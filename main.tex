\documentclass[11pt,a4paper]{report}
\usepackage[utf8]{inputenc}
\usepackage{amsmath}
\usepackage{amsfonts}
\usepackage[french]{babel}
\usepackage{amssymb}
\usepackage{enumerate}

\usepackage[cm]{fullpage}

\usepackage{listings}
\usepackage{xcolor}
\usepackage{verbatim}
\usepackage{framed}
\usepackage{ulem}
\usepackage{pigpen}
\usepackage{pifont}
\usepackage{hyperref}
\usepackage{pbox}

\usepackage{graphicx}
\newcommand{\ens}[1]{\lbrace #1 \rbrace}
\newcommand{\abs}[1]{\vert #1 \vert}

\newcommand{\union}{\cup}
\newcommand{\intersection}{\cap}
\newcommand{\mand}{\wedge}
\newcommand{\mor}{\vee}
 
\newcommand{\bigoh}{\mathcal{O}}

\newcommand{\dessin}[1]{\begin{center}\includegraphics[scale=0.4]{images/#1.png}\end{center}} % 150%
\newcommand{\dessinS}[2]{\begin{center}\includegraphics[scale=#2]{images/#1.png}\end{center}}

\newtheorem{theorem}{Théorème}[section]
\newtheorem{lemma}[theorem]{Lemma}
\newtheorem{proposition}[theorem]{Proposition}
\newtheorem{corollary}[theorem]{Corollary}

\newcommand{\qed}{\nobreak \ifvmode \relax \else
      \ifdim\lastskip<1.5em \hskip-\lastskip
      \hskip1.5em plus0em minus0.5em \fi \nobreak
      \vrule height0.75em width0.5em depth0.25em\fi}


\newenvironment{proof}[1][Preuve]{\begin{trivlist}
\item[\hskip \labelsep {\bfseries #1}]}{\qed\end{trivlist}}
\newenvironment{definition}[1][Définition]{\begin{trivlist}
\item[\hskip \labelsep {\bfseries #1}]}{\end{trivlist}}
\newenvironment{example}[1][Exemple]{\begin{trivlist}
\item[\hskip \labelsep {\bfseries #1}]}{\end{trivlist}}
\newenvironment{remark}[1][Remark]{\begin{trivlist}
\item[\hskip \labelsep {\bfseries #1}]}{\end{trivlist}}


\newcommand{\ey}[1]{E_y\{#1\}}
\newcommand{\els}[1]{E_{LS}\{#1\}}

\newcommand{\yh}{\hat{y}}				% y hat
\newcommand{\xh}{\hat{x}}				% x hat
\newcommand{\fh}{\hat{f}}				% f hat

\newcommand{\vary}[1]{var_y\{#1\}}
\newcommand{\varyx}[1]{var_{y\vert \underline{x}}\{#1\}}
\newcommand{\varls}[1]{var_{LS}\{#1\}}

\newcommand{\pyo}{P(\overline{y})}		% P(y barre)
\newcommand{\yo}{\overline{y}}			% y overline
\newcommand{\ao}{\overline{a}}			% a overline
\newcommand{\xo}{\overline{x}}			% x overline

\newcommand{\ab}{\textbf{a}} % a bold
\newcommand{\cb}{\textbf{c}} % c bold
\newcommand{\wb}{\textbf{w}} % w bold
\newcommand{\yb}{\textbf{y}} % w bold

\newcommand{\exy}{\text{E}_{\underline{x}, y}}
\newcommand{\eyx}[1]{\text{E}_{y \vert \underline{x}} \ens{#1}}
\newcommand{\ex}{\text{E}_{\underline{x}}}
\newcommand{\xu}{\underline{x}}			% x underline
\newcommand{\uu}{\underline{u}}			% u underline

\newcommand{\yens}{y_{\text{ens}}}
\newcommand{\yhens}{\hat{y}_{\text{ens}}}

\newcommand{\LS}{\text{LS}}
\newcommand{\TS}{\text{TS}}
\newcommand{\VS}{\text{VS}}
\newcommand{\GS}{\text{GS}}
\newcommand{\TSE}{\text{TSE}}

\newcommand{\note}[1]{\textit{\textcolor{red}{Note : #1}}}

\title{Synthèse OS}
\author{Jean-Philippe Collette}

\makeindex
\begin{document}	
	\maketitle
	\tableofcontents
	
	\chapter{Introduction}

Un OS est l'intermédiaire entre l'utilisateur et le hardware. Ses deux principales caractéristiques sont d'être juste et performant. Il a pour but

\begin{itemize}
	\item d'exécuter les programmes utilisateurs et rendre leurs problèmes plus faciles
	\item rendre l'ordinateur pratique à utiliser
	\item utiliser de manière efficace le hardware
\end{itemize}

L'OS est entre autre

\begin{itemize}
	\item un allocateur de ressources : il les gère et est l'arbitre de conflits de requêtes en restant efficace et juste (on a les ressources appropriées, ce n'est donc pas nécessaire équitable)
	\item un programme de contrôle, qui prévient des erreurs et de l'utilisation incorrecte d'un ordinateur
\end{itemize}

Le noyau est le seul programme qui tourne tout le temps. Tous les autres programmes sont soit des programmes systèmes, soit un programme applicatif.

	\section{Démarrage d'un ordinateur}
	
	Au lancement de la machine, un programme de boostrap est chargé. Il se trouve initialement dans de la ROM ou de l'EPROM (aussi appelé firmware). Il initialise tous les aspects du système, notamment il charge le noyau et lance son exécution.
	
	Pour des OS récents, il peut arriver que la ROM ne soit pas suffisante pour l'initialisation complète du système. Dès lors, la ROM pointe vers du code stocké sur le disque qui permet de le faire.
	
	\section{Organisation et opérations d'un système}
	
	Tous les périphériques (CPU, contrôleur disque, contrôleur USB, etc) sont connectés à un bus partagé et qui leur donne un accès à de la mémoire partagée.
	
	Il y a ainsi une compétition entre le CPU et les périphériques pour l'utilisation de la mémoire ; ceux-ci s'exécutent de manière concurrente.
	
	Chaque contrôleur de périphérique prend en charge un périphérique particulier et possède un buffer local. Ainsi le CPU peut déplacer des données de/vers la RAM vers/dans ces buffers locaux. Les opérations I/O sont celles qui se produisent du périphérique vers le buffer local du contrôleur. Les contrôleur informent le CPU de la fin de l'opération par une interruption.
	
	\section{Interruptions}
	
	Une interruption transfère le contrôle vers une routine d'interruption en utilisant le vecteur d'interruption, qui contient les adresses de toutes les routines d'interruption.
	
	Lors d'une interruption :
	
	\begin{enumerate}
		\item il y a une sauvegarde du contexte (registres, program counter (PC), etc). Cela est fait en hardware et non en software, sinon on utiliserait un programme qui utilise le PC et les registres.
		\item chargement de l'interrupt handler
		\item désactivation des interruptions, ce qui permet d'éviter de perdre une interruption
		\item exécution du handler
	\end{enumerate}
	
	Un trap est une interruption générée par le logiciel et qui peut être causée par une erreur ou une requête utilisateur.
	
	L'OS est dit interrupt driven car il repose entièrement sur les interruptions pour fonctionner.
	
		\subsection{Gestion d'une interruption}
		
		Ce sont les handler d'interruption qui sauvegardent le PC et les registres plutôt qu'un contrôleur, car ces opérations sont plus rapides en software qu'en hardware, parce que le handler va sauver uniquement les registres utilisés quand du hardware va tout sauver.
		
		\note{Contradiction : la sauvegarde du contexte se fait en hardware ou en software ? Software pour les interruptions et hardware pour un scheduling ?}
	
	Deux types d'interruption peuvent se produire :
	
	\begin{itemize}
		\item vectored interrupt : interruption qui identifie le périphérique qui a déclenché l'interruption.
		\item polling : interruption pour laquelle le handler doit envoyer des signaux à chaque périphérique pour trouver celui qui a déclenché l'interruption.
	\end{itemize}
	
	\section{Structure I/O}
	
	Après qu'une I/O soit démarrée, le contrôle n'est rendu à l'utilisateur que lorsque l'I/O est terminée. Ainsi, une instruction d'attente occupe le CPU jusqu'à la prochaine interruption (souvent une wait loop). Au plus une requête I/O est exécutée à un moment, il n'y a pas de processing d'I/O en même temps.
	
	Le contrôle peut toutefois être retourné à l'utilisateur sans attendre la complétion de l'I/O. Un appel système à l'OS permet à l'utilisateur d'attendre la fin de l'I/O.
	
	Une device-status table contient une entrée pour chaque périphérique I/O et indique son type, son adresse et son état. L'OS utilise cette table pour déterminer l'état d'un périphérique et la modifie pour inclure des interruptions.
	
		\subsection{DMA - Direct Memory Access}
		
		Ce composant est utilisé pour permettre aux périphériques I/O rapides de transmettre de l'information à de la mémoire plus vite. Les données provenant des buffers des contrôleurs peuvent être transférée directement en mémoire sans l'intervention du CPU. Une interruption est générée par bloc, plutôt qu'une interruption par byte.
		
	\section{Structure de stockage}
	
	On a
	
	\begin{itemize}
		\item la mémoire principale, qui peut être accédée par le CPU directement
		\item le stockage secondaire, une extension de la mémoire principale et qui offre une capacité de stockage non volatile
		\item des disques magnétiques, interfacés au système par un contrôleur disque et qui détermine les interactions logiques entre les deux.
	\end{itemize}
	
	Les systèmes de stockage sont organisés en hiérarchie, avec des critères de vitesse, de coût et de volatilité.
		
	\dessin{89}
	
	Plus on est haut dans la hiérarchie et plus on a besoin de composants pour stocker un bit (6 transistors en cache, un transistor et une capacité en RAM). De plus, la consommation d'électricité augmente (la RAM statique (cache) est constamment alimentée).
	
		\subsection{Cache}
		
		Le caching consiste à copier les informations dans des systèmes de stockage plus rapides. Du cache est présent à plusieurs niveaux dans un ordinateur ; des informations sont stockées temporairement d'un stockage lent vers un stockage plus rapide.
		
		Un cache détermine d'abord si l'information désirée est présente. Si c'est le cas, elle est utilisée, sinon elle est copiée dans le cache et utilisée.
		
		Un cache est plus petit que le stockage qui est mis en cache, car la gestion d'un cache est un important problème de design, de même que sa taille et la politique de remplacement.
		
	\section{Architecture ordinateur-système}
	
	La plupart des systèmes utilisent un processeur générique (general-purpose), mais il y en a beaucoup qui possèdent des processus spécifiques (special-purpose).
	
	Les systèmes multi-processeurs (ou systèmes parallèles) sont de plus en plus importants et utilisés et ont les avantages d'être
	\begin{itemize}
		\item plus performants
		\item économe à grande échelle
		\item plus fiable (en terme de dégradation ou de tolérance à la faute)
	\end{itemize}
	
	\dessin{90}
	
	Il y a deux types de systèmes : l'asymetric multiprocessing et le symetric multiprocessing. Exemple d'architecture multiprocessing symétrique :
	
	\dessin{91}
	
	A cause du temps de propagation, il est plus rapide et sûr d'envoyer des données bit à bit qu'en parallèle, car les lignes sont de longueur différentes, ce qui nécessite un mécanisme de synchronisation.
	
	wire $\neq$ transmission (on doit tenir compte de l'endroit où se trouvent les données).
	
	Design dual-core :
		
	\dessinS{92}{0.3}
	
		\subsection{Systèmes distribués/clustered}
		
		Dans ces systèmes, plusieurs ordinateurs travaillent ensembles. Généralement, ils partagent un espace de stockage (SAN, storage-area network) et fournissent des services à haute disponibilité, qui peuvent survivre à des défaillances. Il y a deux types de clustering :
		
		\begin{itemize}
			\item asymetric clustering : une machine est en hot-standby mode
			\item symmetric clustering : plusieurs noeuds font tourner des applications et se monitorent entre elles.
		\end{itemize}
		
		Certains clusters sont designés pour du calcul haute-performance (HPC - high-performance computing), où les applications doivent être écrites pour utiliser de la parallélisation.
		
	\section{Structure d'un OS}
	
	Le multiprogramming est nécessite pour être efficace. En effet, un seul utilisatuer ne peut pas garder le CPU et les périphériques I/O occupés tout le temps. Le multiprogramming organise les jobs (code et données) de façon à ce que le CPU ait toujours quelque chose à exécuter. Ainsi, un ensemble de jobs est gardé en mémoire et l'un d'entre eux est sélectionné et exécuté via du job scheduling. Quand un job doit attendre (par exemple pour une requête I/O), l'OS passe à un autre job.
	
	Le timesharing (ou multitasking) est une extension logique dans laquelle le CPU passe d'un job à l'autre si fréquemment que les utilisateurs peuvent interagir avec tous les jobs en même temps, ce qui donne une interactivité. Le temps de réponse doit ainsi être inférieur à une seconde.
	
	Chaque utilisateur a eu moins un programme en cours d'exécution en mémoire, c'est un processus. Si plusieurs jobs sont prêts à être lancés, on utilise du CPU scheduling. Si les processus ne rentrent pas tous en mémoire, on est les swap. La mémoire virtuelle permet l'exécution de processus qui ne sont pas complètement en mémoire.
	
		\subsection{Opérations d'un OS}
		
		Un OS est basé sur les interruptions lancées par le hardware. Les erreurs logicielles ou les requêtes créent des exceptions ou trap.
		
		L'OS peut se protéger lui-même et les autres composants du système grâce à un dual-mode ; un mode utilisateur et kernel sont définis à travers un bit de hardware. Cela permet de distinguer quand le système fait tourner du code utilisateur ou du code noyau, car certaines instructions demandent des privilèges supplémentaires et sont réservées au noyau. Un appel système change le mode en kernel et un retour remet le mode en utilisateur.
		
		\dessin{93}
		
		Le temps est utilisé pour prévenir une boucle infinie ou un hogging de ressources : une interruption est déclenchée après un certains temps (un compteur géré par l'OS est décrémenté jusqu'à arriver à 0), ce qui déclenche le processus d'ordonnancement, afin de regagner le contrôle ou terminer le programme.
		
		\subsection{Gestion de processus}
		
		Un processus est un programme en cours d'exécution, c'est l'unité de travail dans le système. Le programme est une entité passive, tandis que le processus est une entité active.
		
		Les processus ont besoin de ressources pour accomplir leurs tâches : CPU, mémoire, fichiers, I/O et des données d'initialisation. La terminaison d'un processus nécessite la réclamation des ressources réutilisables.
		
		Un processus à un seul thread ne possède qu'un seul program counter (PC), qui spécifie la position de la prochaine instruction à exécuter. Le processus exécute les instructions séquentiellement, une à la fois, jusqu'à la fin.
		
		Les processus multi-threaded ont par contre un PC par thread.
		
		Typiquement, un système a plusieurs processus, des utilisateurs et des OS qui utilisent un ou plusieurs CPUs en parallèle. La concurrence s'obtient en multiplexant les CPUs aux processus/threads.
		
		L'OS est responsable des activités suivantes :
		
		\begin{itemize}
			\item créer et supprimer les processus des utilisateurs et du système
			\item suspendre et reprendre des processus
			\item fournir des mécanismes pour la synchronisation de processus
			\item fournir des mécanismes pour la communication entre processus
			\item fournir des mécanismes pour gérer les deadlocks
		\end{itemize}
		
\section{Gestion de la mémoire}

Le gestionnaire de mémoire doit déterminer ce qui doit être en mémoire (données et instructions) afin d'optimiser l'utilisation du CPU et les réponses aux utilisateurs. Il a pour activités

\begin{itemize}
	\item le suivi des parties de la mémoire qui sont en train d'être utilisées et par qui
	\item la décision de quels processus (ou parties de processus) et quelles données sont mise en ou hors mémoire
	\item allouer et désallouer de la mémoire quand c'est nécessaire
\end{itemize}

\section{Gestion du stockage}

L'OS fournit une vue logique et uniforme des périphériques de stockage. Les propriétés physiques sont ainsi abstraites en des unités de stockage logiques, les fichiers. Chaque médium est contrôlé par un périphérique et possède des propriétés très variées (temps d'accès, capacité, taux de transfert, méthode d'accès, etc).

Un système de fichiers permet d'organiser les fichiers en répertoires et en gère l'accès (détermine qui peut accéder à quoi).

Les activités de l'OS incluent

\begin{itemize}
	\item la création et la suppression de fichiers et répertoires
	\item des primitives pour manipuler les fichiers et répertoires
	\item un mapping des fichiers dans un stockage secondaire
	\item un backup des fichiers sur un média stable (non volatile).
\end{itemize}

\section{Gestion du stockage de masse}

Généralement un disque est utilisé pour stocker des données qui ne peuvent pas rentrer en mémoire ou des données qui doivent être conservées pour des "longues" périodes de temps. Une bonne gestion de ce stockage est nécessaire, car toute la rapidité d'un ordinateur dépend du sous-système de disques et de ses algorithmes.

Les activités de l'OS sont de

\begin{itemize}
	\item gérer l'espace libre
	\item allouer de l'espace
	\item ordonnancer le disque
\end{itemize}

Certains types de stockage n'ont pas besoin d'être rapide, par exemple les stockages optiques ou magnétiques. Ils doivent cependant toujours être managés. Ils varient du WORM (write-once, read-many-times) et RW (read-write).

\dessinS{94}{0.35}

Les environnements multitâches doivent être prudents et utiliser la valeur la plus récente de variables, quelque soit l'endroit où elles sont stockées dans la hiérarchie.

\dessinS{95}{0.35}

Les environnements multiprocesseurs doivent fournir une cohérence de cache en hardware, de façon à ce que tous les CPUs aient la valeur la plus récente dans leur cache.

Les environnements distribués sont encore plus complexes à gérer, car plusieurs copies d'une même donnée peut exister.

\section{Sous-système I/O}

Un des objectifs de l'OS est de cacher tous les détails des périphériques hardware à l'utilisateur. Un sous-système I/O est responsable de

\begin{itemize}
	\item gérer la mémoire des I/O, ce qui inclut
	\begin{itemize}
		\item le buffering (stocker des données temporairement pendant qu'elles sont transférées)
	\item le cache (stocker des parties de données dans un stockage plus rapide pour améliorer les performances)
	\item et le spooling (le recouvrement de la sortie d'un job avec l'entrée d'autres jobs).
	\end{itemize}
	
	\item l'interface des drivers de périphériques
	\item les drivers pour des périphériques hardware spécifiques.
\end{itemize}

\section{Protection et sécurité}

La protection désigne tous les mécanismes pour contrôler l'accès de processus ou d'utilisateurs à des ressources définies par l'OS.

La sécurité est la défense du système contre des attaques internes ou externes (DOS, vers, virus, vol d'identité, etc).

Les systèmes déterminent généralement, parmi les utilisateurs, qui peut faire quoi. Les utilisateurs ont ainsi un identifiant unique qui peut être associé à des fichiers et des processus. Des groupes (identifiés aussi par des identifiants uniques) permettent à des ensembles d'utilisateurs d'être définis et de les associer à des processus ou des fichiers.

Une escalade des privilèges (privilege escalation) permet à un utilisateur de changer d'ID effectif pour posséder plus de droits.

\section{OS Open source}

Ce sont des OS dont le code source est disponible, à l'opposé des OS dont on n'a que des binaires (closed-source).
	
	\chapter{Structure du système}

\section{Services des OS}

Un ensemble de services est fourni à l'utilisateur par l'OS :

\begin{itemize}
	\item une interface utilisateur, qui vont de CLI (command-line interface) à GUI (Graphics User Interface).
	\item l'exécution de programmes ; un OS peut charger un programme en mémoire, le lancer et le terminer (normalement ou après une erreur).
	\item des opérations I/O, qui impliquent un fichier ou un périphérique I/O.
	\item la manipulation d'un système de fichier : il faut pouvoir lire, écrire et supprimer des fichiers et des répertoire, ainsi qu'effectuer des recherches, gérer les permissions, etc.
	\item des moyens de communication entre les processus, sur le même ordinateur ou à travers le réseau. Cela peut se faire par de la mémoire partagée ou des envois de messages
	\item la détection d'erreurs, dans le CPU, dans les périphériques I/O ou dans les programmes utilisateurs. Pour chaque erreur, l'OS doit effectuer l'action appropriée pour la corriger au mieux. Des facilités de debug permettent d'utiliser le système plus efficacement.
	
	\item l'allocation de ressources, lorsque plusieurs utilisateurs ou jobs concurrents en demandent. Certains types de ressource (comme les cycles CPU, la mémoire principale, etc) peuvent avoir du code d'allocation spécifique, les autres (les périphériques I/O par exemple) ont un code général de demande et de libération.
	\item la gestion de comptes utilisateurs, pour savoir qui consomme quelle quantité de ressources et quel genre
	\item la protection et la sécurité, afin notamment que des processus concurrents n'interfèrent pas. La protection implique que tous les accès aux ressources sont contrôlés, tandis que la sécurité du système pour des outsiders nécessite une authentification utilisateur et est étendu à la défense de périphériques I/O externes contre des tentatives d'accès invalides.
\end{itemize}

\dessin{1}


\section{Interfaces}

	\subsection{CLI}
	
	Une CLI permet d'entrer directement les commandes. Parfois c'est implémenté dans le noyau (ex : DOS), parfois avec des programmes systèmes (ex : UNIX). Il existe de nombreuses variantes de CLI (les shells), avec des ajous de commandes ou d'autres fonctionnalités. 
	
	\subsection{GUI}
	
	Interface plus user-friendly, avec le support de la souris.
	
\section{Appels systèmes}

L'OS fournit une interface de programmation, généralement disponible sous la forme d'API. Il s'agit d'une abstraction des appels systèmes, qui s'utilisent comme des fonctions mais qui s'exécutent différemment. Généralement un langage de haut niveau est utilisé (C, C++).

Raisons d'utiliser une API plutôt que des appels systèmes :

\begin{itemize}
	\item les appels systèmes diffèrent d'une plate-forme à l'autre ; une API permet une migration plus simple.
	\item il peut y avoir des évolutions côté API et appels systèmes sans qu'il ne soit nécessaire de changer le code d'appel.
	\item une API fournit plus de fonctionnalités que les appels systèmes.
	\item l'API supporte plusieurs versions de l'OS et peut s'y adapter.
\end{itemize}

Chaque appel système est associé à un nombre ; l'interface des appels systèmes maintient une table qui les utilisent pour les indexer. Cette interface va appeler le syscall dans le noyau de l'OS et retourner le statut du syscall et une éventuelle valeur de retour. L'utilisateur n'a pas besoin de connaître l'implémentation du syscall, il n'a qu'à suivre l'interface et comprendre ce que l'OS renvoie.

Par exemple, un appel à printf() en C appelle le syscall write().

\dessin{2}

	\subsection{Passage de paramètres}
	
	Il y a en général trois manières de passer des paramètres à un syscall :
	
	\begin{enumerate}
		\item passer les paramètres par les registres, mais cela ne va pas dans le cas où il y a plus de paramètres que de registres ou si les valeurs sont trop grandes pour les stocker dedans. Cependant, c'est la méthode la plus rapide.
		\item les paramètres sont stockés en mémoire dans un bloc ou une table, et une adresse du bloc est passée comme paramètre dans un registre (Linux, Solaris)
		\item les paramètres sont placés (ou pushed) sur la stack par le programme et popped par l'OS
	\end{enumerate}
	
	Grâce aux deux dernières méthodes, il n'y a pas de limite sur le nombre ou la taille des arguments passés.
	
	Exemple de MS-DOS :
	
	\dessin{4}
	
	Lors de l'exécution d'un programme, une partie de l'interpréteur est libérée car à cette époque la mémoire est restreinte et une seule tâche est exécutée à la fois. C'est en effet inutile d'avoir l'interpréteur et le programme en mémoire en même temps.
	
	Exemple de Solaris :
	
	\dessinS{5}{0.3}
	
	\subsection{Types d'appels systèmes}
	
	\dessin{3}
	
\section{Programmes systèmes}

Ce sont des programmes qui fournissent un environnement pour le développement et l'exécution de programmes : 

\begin{itemize}
	\item manipulation de fichiers
	\item informations de statut : on peut récupérer des infos sur le système (heure, date, mémoire disponible, etc), ainsi que des infos de login, de débuggage et de performance. Ces programmes affichent généralement une sortie dans le terminal, d'autres implémentent cela dans un registre, qui est utilisé pour stocker et récupérer des infos de configuration.
	\item modification de fichiers.
	\item support de langages de programmation : compilateur, assembleurs, débuggers et interpréteurs.
	\item chargement et exécution de programmes.
	\item communication  : connections virtuelles entre des processus, des utilisateurs et d'autres systèmes informatiques.
	\item programmes applicatifs.
\end{itemize}


\section{Design et implémentation d'un OS}

La structure interne des OS varie très fort. On commence par définir les buts et les spécifications, qui sont conditionnés par le choix du hardware et du type de système. On distingue les buts des utilisateurs (l'OS doit être convénient à utiliser, à aborder, fiable, sûr et rapide) des buts du système (facile à designer, à implémenter et à maintenir, flexible, fiable, sans erreurs et efficace).

Un principe important est de séparer les polices (déterminent ce qui sera fait) des mécanismes (détermine comme faire quelque chose), ce qui permet une grande flexibilité si les décisions de police doivent changer.

	\subsection{Approche par couche}
	
	Exemple de MS-DOS : il est écrit pour être le plus fonctionnel tout en tenant sur un espace restreint. Il n'est pas divisé en modules, et même s'il y a une structure, il n'y a pas une distinction claire entre les interfaces et les niveaux de fonctionnalités.
	
	\dessinS{6}{0.3}
	
	Les OS sont divisés en couches, chacune étant construite au-dessus d'autres. La couche la plus basse (layer 0) est le hardware, tandis que celle qui est la plus haute (layer $N$) est l'interface utilisateur. Pour être modulaires, les couches sont sélectionnées de manière à ce que chacune utilise des fonctions et services de couches inférieures. Plus on descend dans les couches, plus l'abstraction est faible. Du coup, il n'y a pas d'intérêt, pour une couche $c$, d'accéder à la couche $c + 1$. Chacune couche a un niveau de détails différent.
	
	Exemple de UNIX : initialement, il n'y a que deux parties : les programmes systèmes et le noyau (tout entre l'interface de syscall et le hardware, et qui fournit beaucoup de fonctions pour un seul niveau).
	
	\dessin{7}
	
	Avoir beaucoup de couches n'a pas que des avantages :
	
	\begin{itemize}
		\item[+] cela devient de plus en plus simple
		\item[-] il y a un overhead pour chaque couche
	\end{itemize}
	
	Ne pas avoir beaucoup de couches n'est pas non plus souhaitable :
	
	\begin{itemize}
		\item[+] rapide
		\item[-] moins modulaire car il y a beaucoup de code dans chaque couche
	\end{itemize}
	
	
	
	\subsection{Structure de microkernel}
	
	Le microkernel se place aussi loin que possible du noyau, dans l'espace utilisateur. Les communications entre les modules utilisateurs se fait avec des envois de messages.
	
	\begin{itemize}
		\item[+] il est plus facile d'étendre un microkernel
		\item[+] il est plus facile de porter l'OS sur de nouvelles architectures
		\item[+] c'est plus fiable (moins de code tourne en mode kernel)
		\item[+] plus sécurisé
		\item[-] overhead pour les communications entre l'espace utilisateur et l'espace kernel
	\end{itemize}
	
	Exemple de Mac OS X :
	
	\dessinS{8}{0.3}
	
	\subsection{Approche à base de modules}
	
	La plupart des OS modernes utilisent des modules kernel. C'est une approche orientée-objet, où chaque composant est séparé. Chacun dialogue avec les autres à travers des interfaces et est chargé à la demande dans le noyau. C'est similaire à des couches tout en étant plus flexible.
	
	Exemple : Solaris.
	
\section{Débuggage}

Les OS peuvent générer des fichiers de log qui contiennent les informations d'erreurs. Ainsi :

\begin{itemize}
	\item la défaillance d'une application peut générer un core dump, un fichier qui capture la mémoire du processus
	\item la défaillance de l'OS génère un crash dump, un fichier qui contient la mémoire du kernel
\end{itemize}

Loi de Kernighan : "Debugging is twice as hard as writing the code in the first place. Therefore, if you write the code as cleverly as possible, you are, by definition, not smart enough to debug it."

L'outil DTrace (Solaris, FreeBSD, Mac OS X) permet d'utiliser des sondes lorsque du code est exécuté, de capturer l'état des données et le récupérer.


\section{Génération d'OS}

Les OS sont conçus pour tourner sur n'importe quelle classe de machine, pour autant que le système soit configuré. Ils doivent cependant être accessibles au hardware pour qu'il puisse le lancer.

Le boot (booting) démarre un ordinateur en chargeant le kernel en lisant à un endroit fixe de la mémoire. Le noyau se trouve dans le MBR (master boot record) s'il peut rentrer dedans, sinon le bootloader est placé dedans, qui lui va chercher le kernel où il faut.



	
	\chapter{Les processus}

Un OS peut faire tourner une variété de programmes. On a des systèmes batch où les programmes, appelés jobs, sont lancés sans aucune autre intervention, et les systèmes time-shared (ou online), qui sont eux interactifs.

Un processus est un programme en cours d'exécution, de manière séquentielle. Il inclut

\begin{itemize}
	\item un PC (programme counter), qui permet de savoir où en est l'exécution. Dans une machine, il y a un PC par core (ou plus si les cores sur multipipelinés).
	\item une pile (stack), afin de permettre des appels de fonctions (pour sauver les arguments, placer les arguments et mettre l'adresse de retour). La partie du stack lié à un appel de fonction est la stack frame.
	\item une section de données
\end{itemize}

\dessinS{9}{0.3}

Un processus peut avoir différents états :
\begin{itemize}
	\item new : le processus est en train d'être créé
	\item running : les instructions sont en train d'être exécutées
	\item waiting : le processus attend qu'un évènement se produise
	\item ready : le processus attend l'usage du processeur
	\item terminated : le processus a fini son exécution
\end{itemize}

\dessin{10}

Toutes les informations associées à un processus sont stockées dans un PCB (process control block). Il permet de savoir ce qu'utilise le processus, notamment lorsqu'il se termine. Sont stockés :

\begin{itemize}
	\item l'état du processus
	\item le program counter
	\item les registres du CPU
	\item les infos de scheduling du CPU
	\item les infos de gestion de mémoire
	\item les infos de compte
	\item les infos de status I/O
\end{itemize}

Toutes ces informations sont stockées/chargées lorsqu'un processus perd/obtient l'accès au CPU, dans le cadre d'un changement de contexte. ce changement est hardwired et donc optimisé pour que cela se fasse en un minimum de cycles.

Il y a plusieurs listes de processus dans le noyau ; ceux-ci passe d'une liste à l'autre en fonction de leur évolution :

\begin{itemize}
	\item job queue : l'ensemble de tous les processus du système
	\item ready queue : l'ensemble des processus résidant en mémoire, prêts et en attente d'être exécutés
	\item device queues : l'ensemble des processus attendant un périphérique I/O. Il y a une file par device.
\end{itemize}

\dessin{11}

\section{Schedulers}

Il y a deux schedulers :

\begin{itemize}
	\item long-term scheduler (ou job scheduler) : sélectionne les processus qui seront placés dans la ready queue. Il est invoqué irrégulièrement (secondes, minutes) (donc peut se permettre d'être lent) et contrôle le degré de multiprogramming.
	\item short-term scheduler (ou CPU scheduler): sélectionne quels processus doivent être exécutés et alloue le CPU. Il est appelés très fréquemment (millisecondes), donc doit être très rapide et efficace.
\end{itemize}

Dans ce contexte, on peut définir le jiffy comme le temps entre deux interruptions d'horloge. Il est en général de 10ms (4ms sous Linux).

Certains OS n'ont pas de scheduler à long terme. Du coup, ils gardent tout en mémoire ; il y a donc une limite sur le nombre de processus résidant en mémoire.

\note{$\rightarrow$ pourquoi cela est-il acceptable ?}

\dessin{12}

S'il y a trop de processus et qu'il y a du swap (des PCBs et de toute la mémoire associée), tout l'OS va ralentir car le disque est lent.

Les processus peuvent être limités de deux façons différentes :

\begin{itemize}
	\item I/O-bound : ils passent plus de temps à des opérations d'I/O qu'à des calculs, ce qui donne beaucoup de petits CPU bursts
	\item CPU-bound : ils passent plus de temps à faire des calculs, ce qui donne des long CPU bursts.
\end{itemize}

Idéalement, il faut un mix d'IO-bound (si tous les processus l'étaient, la ready queue serait souvent vide) et de CPU-bound (la ready queue serait importante) pour une bonne utilisation des ressources. Toutefois, pour un système, on préfèrera des processus I/O-bound, car des processus CPU-bound ont tendance à réduire l'interactivité.


\section{Création de processus}

Des processus parents créent des processus enfants, qui à leur tour créent des processus, ce qui forme un arbre. Chaque processus est identifié par un identifiant, le PID.

Il y a un partage des ressources : les enfants se partagent un sous-ensemble des ressources du parent.

\note{$\rightarrow$ Parent and children share all resources but parent and child share no resources (? [3.18])}

Au niveau de l'exécution, les enfants et le parent s'exécutent simultanément, mais le parent attend que ses enfants se terminent avant de se terminer lui-même.

Lors de la création d'un enfant, l'address space du parent est dupliqué et l'enfant exécute un programme qui est chargé dedans. Sous UNIX, fork crée un nouveau processus et exec remplace l'espace mémoire du processus par un nouveau programme.

\dessin{13}

\section{Terminaison d'un processus}

Une fois qu'un processus a exécuté sa dernière instruction, il va demander à l'OS de le supprimer (exit). Les données de sortie sont transmises au parent (via wait), ensuite les ressources du processus sont désallouées par l'OS.

Les parent peuvent terminer l'exécution de leurs enfants avec abort, par exemple s'ils dépassent la quantité de ressources allouées ou s'ils n'ont plus de raison d'être.

Si un parent se termine, certains OS ne permettent pas que les enfants continuent de tourner, du coup ils sont également terminés. Sinon, les enfants deviennent des processus zombies. Le problème dans ce cas est qu'ils sont hors de l'arbre des processus tout en consommant des ressources.

\section{Communication entre les processus}

Les processus d'un système peuvent être indépendants ou coopérer. Dans ce cas, un processus peut affecter ou être affecté par les autres, notamment au niveau de la mémoire partagée. Il y a plusieurs raisons à l'existence d'une coopération :

\begin{itemize}
	\item partage d'informations
	\item speedup des calculs
	\item modularité
	\item par que c'est plus convénient
\end{itemize}

Des processus coopérants ont besoin de communication interprocessus (inteprocess communication, IPC). Il y a deux modèles différents d'IPC : (a) l'envoi de messages et (b) la mémoire partagée.

\dessin{14}

	\subsubsection{Envoi de messages}
	
	C'est un mécanisme qui permet de ne pas devoir gérer des variables partagées. Deux opérations sont possibles : send et receive.
	
	Si deux processus souhaitent communiquer, ils doivent établir un lien de communication entre eux et ensuite s'échanger des messages avec send et receive.
	
	L'implémentation d'un tel moyen de communication peut être physique (mémoire partagée, bus) ou logique (propriétés logiques).
	
	Beaucoup de questions surviennent pour l'implémentation de ces liens :
	
	\begin{itemize}
		\item How are links established?
		\item Can a link be associated with more than two processes?
		\item How many links can there be between every pair of communicating processes?
		\item What is the capacity of a link?
		\item Is the size of a message that the link can accommodate fixed or variable?
		\item Is a link unidirectional or bi-directional?
	\end{itemize}

	Il existe deux types de communication : direct et indirect.
	
	\paragraph{Communication directe} Les processus doivent se nommer explicitement : send(P, message) et receive(Q, message). Les propriétés du lien de communication sont que
	
	\begin{itemize}
		\item le lien est établi automatiquement
		\item un lien est associé à exactement une paire de processus communicant
		\item entre chaque paire, il n'y a qu'un et un seul lien
		\item le lien peut être unidirectionnel, mais généralement il est bi-directionnel
	\end{itemize}
	
	\paragraph{Communication indirecte} Les messages sont envoyés et reçus à partir de mailboxes (connus sous le nom de port). Chaque mailbox a un identifiant unique et les processus ne peuvent communiquer que s'ils partagent une mailbox. Le lien de communication a les propriétés suivantes :
	
	\begin{itemize}
		\item le lien est établi uniquement si les processus partagent une mailbox
		\item un lien peut associer plusieurs processus
		\item chaque paire de processus peut avoir plusieurs liens de communication
		\item les liens peuvent être unidirectionnels ou bidirectionnels.
	\end{itemize}
	
	On a comme opérations la création d'une mailbox, l'envoi et la réception de messages à travers une mailbox et la destruction de la mailbox. Les primitives sont alors send(A, message) et receive(A, message), où A est une mailbox.
	
	Supposons que $P_1$, $P_2$ et $P_3$ partagent une mailbox $A$. $P_1$ envoie un message et $P_2$ et $P_3$ appellent receive. Qui récupère le message ? Diverses solutions existent :
	
	\begin{itemize}
		\item restreindre un lien à l'association d'au plus 2 processus
		\item permettre à uniquement un seul processus d'effectuer l'opération de réception
		\item permettre au système de choisir arbitrairement le récepteur. L'expéditeur sera alors informé de qui a reçu le message.
	\end{itemize}
	
	
	\subsubsection{Synchronisation}
	
	L'envoi de messages peut être bloquant ou non-bloquant.
	
	Bloquer l'envoi de message est considéré comme synchrone : l'expéditeur est bloqué jusqu'à ce que le message soit reçu et inversement le récepteur est bloqué tant que le message n'a pas été envoyé.
	
	Ne pas bloquer l'envoi de message est considéré comme asynchrone : l'expéditeur envoie le message et continue son exécution, tandis que le récepteur obtient un message valide ou null.
	
	\subsubsection{Buffering}
	
	Une file de message est attachée à chaque lien. Trois politiques sont possibles :
	
	\begin{itemize}
		\item capacité 0 : aucun message. L'expéditeur doit alors attendre le récepteur (rendezvous)
		\item capacité limitée : au maximum $n$ messages peuvent être envoyés, l'expéditeur doit alors attendre si le lien est plein.
		\item capacité illimitée : l'expéditeur n'attend jamais.
	\end{itemize}
	
	\chapter{Threads}

\dessinS{15}{0.3}

Lorsqu'il y a un fork, il faut copier la mémoire du parent dans celle de l'enfant ; le plus simple est de dupliquer toute la table des pages. Pour augmenter les performances et ne pas tout recopier, la table du parent va être partagée avec celle de l'enfant. Les pages ne seront dupliquées que quand l'enfant voudra effectuer des écritures. Les lectures quant à elles ne nécessitent pas de copie.

Quand un processus possède des threads, il a des tâches qui partagent sa mémoire. Chacun des threads a ses propres registres, stack et fil d'exécution. Créer un thread est beaucoup plus facile que créer un processus car il est beaucoup plus léger (il ne possède pas de liste de thread et ne doit pas s'insérer dans l'arbre des processus) ; la magnitude entre créer un processus et un thread est de l'ordre de 10. Par contre, un thread est toujours associé à un processus.

Les bénéfices du multithreading :

\begin{itemize}
	\item capacité de réponse
	\item partage des ressources
	\item économie
	\item scalabilité
\end{itemize}

Programmer un système multicoeur est beaucoup plus difficile pour les programmeurs : il faut diviser les activités, les balancer, diviser les données et tenir compte d'une éventuelle dépendance, et tester et débugger les programmes.

L'hyperthreading d'un core est la duplication du datapath et du controlpath : seul le cache et l'ALU sont partagés. Pour ce dernier, les registres sont dupliqués et alternés lors de l'exécution. Dès lors, il y a un bon match entre l'hyperthreading et le multithreading, on peut même faire tourner deux threads sur le même core. C'est meilleur qu'avec plusieurs processus (multiprocessing)  car le cache hit rate est plus grand et parce que deux processus ont des espaces d'adresses différents.

La gestion des threads se fait par des librairies au niveau utilisateur (POSIX avec Pthread, Win32 threads et Java threads). Le noyau aussi utilise des threads, par exemple pour prendre en charge les appels systèmes des utilisateurs.

\section{Modèles de multithreading}

	Quand un thread utilisateur fait appel à un service du noyau, il doit être pris en charge par un thread kernel. Plusieurs modèles existent.

	\subsection{Many-to-One}
	
	Plusieurs threads users sont liées à un seul thread kernel (Solaris Green threads, GNU Portable Threads). Le problème est qu'on n'a qu'un seul thread qui tourne à la fois (s'il y a plus d'un thread utilisateur, pendant qu'un tourne, les autres attendent), donc c'est très peu performant.

	\dessinS{16}{0.3}
	
	\subsection{One-to-One}
	
	Chaque thread user est mappé à un thread kernel (Windows, Linus, Solaris 9).
	
	\dessinS{17}{0.3}
	
	Le problème de la concurrence de threads utilisateurs est réglé, mais ici créer un thread en crée en fait deux. Vu que chaque thread consomme des ressources, il y a une limite dans le noyau (par un manque de descripteurs notamment). De plus, les threads kernel ne tournent pas tout le temps (seulement quand le thread utilisateur associé effectue un syscall), du coup des ressources sont réservées pour rien.
	
	\subsection{Many-to-Many}
	
	Plusieurs threads utilisateurs peuvent être mappés à plusieurs threads kernel (Solaris $>$ 9, Windows NT). Cela permet à l'OS de créer un nombre suffisant de threads kernel.
	
	\dessinS{18}{0.3}
	
	\subsection{Modèle à deux niveaux}
	
	C'est similaire au modèle many-to-many, si ce n'est qu'on permet à un thread user d'être lié à un thread kernel.
	
	\dessinS{19}{0.3}
	
\section{Librairies de thread}

Des librairies fournissent une API permettant de créer et gérer des threads. Il y a deux manières de les implémenter :

\begin{itemize}
	\item une librairie entièrement dans l'espace utilisateur
	\item une librairie au niveau du kernel supportée par l'OS
\end{itemize}

Pthreads est une librairie qui donne des threads au niveau utilisateur et kernel, avec une API standardisée POSIX. Elle est présente sous les systèmes UNIX (Solaris, Linux, Mac OS X).

Les threads Java sont supportés par la machine virtuelle, qui généralement utilise le même modèle que ce que l'OS utilise.

\section{Problèmes de threading}

	\subsection{Sémantiques de fork et exec}
	
	Un thread qui appelle fork() peut soit créer un nouveau processus avec tous les threads, soit créer un nouveau thread équivalent.
	
	Lorsqu'un thread fait exec(), le code du processus parent est changé, donc tous les threads sont supprimés.
	
	\subsection{Annulation de threads}
	
	Supposons que l'on veuille terminer un thread avant qu'il ait terminé. Il y a en général deux approches :
	
	\begin{itemize}
		\item une annulation asynchrone, qui termine le thread immédiatement. Le problème est qu'il pourrait y avoir une réservation de ressources (par exemple un fichier pourrait rester ouvert avec la mort d'un thread).
		\item une annulation déférée, où le thread vérifie périodiquement s'il doit s'arrêter
	\end{itemize}
	
	
	\subsection{Gestion des signaux}
	
	Les signaux sont utilisés dans les systèmes UNIX pour notifier un processus qu'un évènement particulier s'est produit. Ainsi, un signal handler est utilisé pour traiter les signaux : le signal est généré par un évènement particulier, il est délivré à un processus et est traité.
	
	Pour les threads, plusieurs options existent :
	
	\begin{itemize}
		\item délivrer le signal à tous les threads pour lesquels le signal s'applique
		\item délivrer le signal à tous les threads du processus
		\item délivrer le signal à certains threads
		\item assigner à un thread la tâche de recevoir tous les signaux
	\end{itemize}
	
	\subsection{Thread Pools}
	
	Cette technique consiste à créer un certain nombre de threads, qui sont stockés dans une pool en attendant d'être assignés à une tâche. Cela a plusieurs avantages :
	
	\begin{itemize}
		\item c'est légèrement plus rapide de servir une requête avec un thread existant que d'en créer un nouveau
		\item le nombre de threads pour l'application peut être limité à la taille de la pool.
	\end{itemize}
	
	\subsection{Thread spécifique aux données}
	
	On permet à chaque thread d'avoir sa propre copie des données. Cela est utile quand on n'a pas le contrôle sur la création du thread, notamment dans le cas d'une thread pool.
	
	\subsection{Activation par un scheduler}
	
	Les modèles many-to-many et à deux niveaux nécessitent un moyen de communication pour maintenir le nombre approprié de threads alloués à une application. Ainsi, les scheduler activations proposent des upcalls, un mécanisme de communication du kernel vers la librairie de thread, et qui permet de maintenir ce nombre de threads.
	
	\chapter{Scheduling de processus}

Une utilisation maximale du CPU est obtenue avec du multiprogramming. On a alors des cycles de burst CPU-I/O ; l'exécution d'un processus consiste en un cycle d'exécution CPU et d'attente I/O. On a alors tout intérêt à passer le CPU à un autre processus.

On peut tracer la distribution des bursts et remarquer que la majorité dure peu de temps.

\dessin{20}

La préemption est la possibilité de retirer une ressource qui est toujours en train d'être utilisée.

L'ordonnanceur du CPU sélectionne un processus parmi ceux qui sont en mémoire et qui sont prêts à être exécutés et va lui allouer le CPU. Le scheduler prend une décision lorsqu'un processus

\begin{enumerate}
	\item passe de l'état running à l'état wait
	\item[$\rightarrow$] Non préemptif, car le CPU doit être switché sinon il n'est pas utilisé
	\item passe de l'état running à l'état ready (lorsqu'il y a une interruption)
	\item[$\rightarrow$] Préemptif
	\item passe de l'état wait à ready
	\item[$\rightarrow$] Préemptif, il faut préempter le CPU pour le processus qui est prêt
	\item se termine
	\item[$\rightarrow$] Idem que 1
\end{enumerate}


Un système est dit work conserving s'il alloue des ressources lorsqu'il y a une demande. Dès lors, un système non-work conserving peut ne pas attribuer des ressources à un processus qui en a besoin.

Un accès mémoire ne met par le processus en état d'attente, car le délai est très court (quelques dizaines/centaines de cycles). L'overhead du changement de contexte est trop gros comparé à ce temps d'attente, qui lui-même est beaucoup plus petit que pour une opération I/O. C'est du non work conserving à petite échelle.

De plus, une mise en attente à chaque accès mémoire n'est pas envisageable, car il y aurait un changement de contexte quasiment à chaque instruction. Cela va entraîner une diminution de l'efficacité du cache, car beaucoup de processus sont exécutés en même temps. Dès lors, il y aura une partition du cache, plus de cache miss et donc plus d'accès à la mémoire. Au final, tout le système est ralenti.

Dans un GPU, il y a des milliers/millions de threads. Ils sont mis en attente lors d'un accès mémoire en espérant que d'autres threads soient prêts avec leurs données. Un context switch est très rapide (1-2 coups d'horloge) par rapport à un CPU.

Le dispatcher est un module qui donne concrètement le contrôle du CPU au processus sélectionné par l'ordonnanceur à court terme. Cela implique

\begin{enumerate}
	\item un changement de contexte
	\item un switch vers le mode utilisateur
	\item un saut à la bonne position dans le programme de l'utilisateur pour le reprendre
\end{enumerate}

On ne peut pas mélanger ces étapes, car (1) le context switch doit être exécuté en mode admin et (2) le PC doit être changé après le context switch pour ne pas perdre le fil d'exécution. L'étape 2 doit être la dernière étape si le dispatcher est software, car lui-même utilise le PC.

La latence de dispatch est le temps pris par le dispatcher pour arrêter un processus et en démarrer un autre, et fait partie du context switch latency (qui compte aussi le scheduling latency).

\section{Critères d'ordonnancement}

\begin{enumerate}
	\item utilisation du CPU : il faut garder le CPU aussi utilisé que possible
	\item le débit de processus qui complètent leur exécution par unité de temps
	\item le turnaround time : la quantité de temps pour exécuter un processus particulier
	\item le waiting time : la quantité de temps qu'un processus doit attendre dans la ready queue
	\item le response time : la quantité de temps qui s'écoule entre l'émission de la requête jusqu'à ce que la première réponse soit produite (et non sortie, pour tenir compte des environnement de type time-sharing).
\end{enumerate}

Les deux premiers sont situés au niveau du système et sont à maximiser, tandis que les trois autres sont liés aux processus et doivent être minimisés. Ces deux perspectives doivent être prises en compte ; par exemple, s'il n'y a qu'un seul processus, les deux premiers critères ne seront pas respectés mais bien les trois autres. Il y a un tradeoff entre l'usage des ressources par un processus individuel et la mise à disposition de cette ressource pour tous les autres processus.

\section{Scheduling de processus}

\subsection{FIFO Scheduling}

Exemple :

\dessinS{21}{.3}

Les performances dépendent de l'ordre d'arrivée des processus. Par exemple, si on a $P_2$, $P_3$ et $P_1$, les temps d'attente sont 6, 0 et 3 (en moyenne 3).

Effet Convoy : des processus longs peuvent ralentir tout le système.

\subsection{Shortest-job-first scheduling}

On associe à chaque processus la longueur de son prochain CPU burst. On utilise alors ces longueurs pour ordonnancer les processus qui dureront le moins longtemps.

Cette stratégie est optimale, car elle donne le temps d'attente moyen minimal. On ne peut cependant pas prédire exactement cette durée car on ne peut voir dans le futur, mais on peut toujours se baser sur le passé des processus et utiliser une moyenne exponentielle.

Si $t_n$ est la longueur du $n$ème burst CPU, $\tau_{n+1}$ la valeur prédite du prochain burst et si $0 \leq \alpha \leq 1$, alors

$$\tau_{n + 1} = \alpha t_n + (1 - \alpha) \tau_n$$

Si $\alpha = 0$, l'histoire récente n'est pas utilisée. Si $\alpha = 1$, seul le dernier burst compte.

\subsection{Scheduling avec priorité}

Une priorité (généralement un entier) est associée à chaque processus, et le CPU est alloué au processus qui a la plus haute priorité (le plus petit entier).

\note{La liste qui suit vient de mes notes, je ne sais pas d'où ça vient.}
\begin{itemize}
	\item nouvelle façon de préempter un processus : ajout d'un processus avec une plus petite priorité
	\item non préemptive : processus avec une plus grande priorité
\end{itemize}


La stratégie SJF est un scheduling avec priorité où la priorité est la prédiction de la durée du prochain burst CPU.

Problème : starvation ; les processus de basse priorité peuvent ne jamais être exécutées.

Solution : aging ; au fur et à mesure que le temps s'écoule, la priorité des processus augmente.

\subsection{Round Robin}

Chaque processus possède une unité de temps CPU (time quantum), généralement entre 10 et 100 ms. Après que ce temps soit écoulé, le processus est préempté et ajouté à la fin de la ready queue.

S'il y a $n$ processus dans la ready queue et que le quantum est $q$, chaque processus possède $\frac{1}{n}$ du temps CPU dans des morceaux d'au plus $q$ unités de temps à la fois. Les processus n'attendent jamais plus de $(n - 1)q$ unités de temps.

Plus $q$ est petit et plus il y a d'interactions. Cependant,

\begin{itemize}
	\item si $q$ est grand, on retourne dans le cas d'une FIFO
	\item si $q$ est petit, il faut qu'il soit tout de même suffisamment large par rapport au context switch, sinon l'overhead qu'il introduit (quelques micro-secondes) sera trop gros.
\end{itemize}

\subsection{File à plusieurs niveaux}

On partitionne  la ready queue en deux files séparées avec chacune son propre algorithme d'ordonnancement :

\begin{itemize}
	\item foreground, pour les processus interactifs, avec un round robin
	\item background, pour les processus batch, avec une file FIFO
\end{itemize}

Un ordonnancement doit alors être pratiqué entre les deux files :

\begin{itemize}
	\item avec un scheduling à priorité fixe (par exemple servir tous les processus foreground puis background), il y a un risque de starvation (la file background pourrait ne jamais être traitée)
	\item avec du time slicing, chaque file a une certaine quantité du temps CPU, qui est ensuite redistribué aux processus (par exemple 80\% pour la file foreground, 20\% pour la file background).
\end{itemize}

De cette façon, on peut construire plusieurs files avec plusieurs niveaux de priorité.

\subsection{File à plusieurs niveaux avec feedback}

Un processus peut changer de file (implémentation d'un aging). Une file de ce type est définie par

\begin{itemize}
	\item le nombre de files
	\item l'algorithme de scheduling de chaque file
	\item la méthode utilisée pour déterminer quand un processus est upgradé
	\item la méthode utilisée pour déterminer quand un processus est dégradé
	\item la méthode utilisée pour déterminer dans quelle file un processus va entrer quand il nécessite un service
\end{itemize}

Exemple :

\dessinS{22}{0.3}

\section{Scheduling de threads}

Une distinction doit être faite entre

\begin{itemize}
	\item les threads kernel, qui sont ordonnancés directement. Il y a une compétition entre tous les threads dans le système : system-contention scope (SCS).
	\item les threads utilisateurs, qui sont ordonnancés à travers la librairie de threads ; les processus qui les contiennent sont ordonnancés. Il y a une compétition dans les processus : process-contention scope (PCS).
	
	Si le système ordonnançait directement les threads des utilisateurs, il y a un risque d'inégalité. Par exemple, si un utilisateur $U1$ crée 99 threads et un autre utilisateur $U2$ en crée 1, $U2$ aura 1\% de l'usage du CPU.
\end{itemize}

\section{Scheduling multi-processeur}

Le scheduling est beaucoup plus compliqué quand plusieurs CPUs (généralement homogènes) sont disponibles. On distingue

\begin{itemize}
	\item l'asymmetric multiprocessing : un seul processeur est réservé pour le noyau, tandis que les autres sont dédiés aux processus utilisateurs. L'avantage est que le kernel ne gère pas le multiprocessing (lock, sémaphores, etc), mais il y a un risque de bottleneck
	
	\item le symmetric multiprocessing (SMP) : chaque processus s'auto-ordonnance. Les processus sont soit rassemblés dans une ready queue commune, soit il y a une ready queue pour chaque processeur (mais risque de files vides, donc de mauvais load balancing même s'il est simplifié. De plus, il faut un mécanisme de synchronisation).
\end{itemize}

L'affinité d'un processus à un processeur est le fait qu'un processus s'exécute toujours sur le même processeur, notamment pour profiter de la cache, mais aussi parce que la distance physique entre le CPU utilisé par le processus et la mémoire est plus grande s'il y a un changement. Cela évite un risque de NUMA (Non Uniform Memory Access).

\dessin{23}

Il y a plusieurs types d'affinité :

\begin{itemize}
	\item soft affinity : selon certaines circonstances (par exemple, trop de processeurs), on autorise un changement de processeur.
	\item hard affinity : aucun changement n'est permis.
\end{itemize}

\section{Processeurs multi-core}

Il s'agit de processeurs avec plusieurs coeurs sur une seule puce. Cela permet d'avoir de meilleurs performances et de consommer moins d'énergies.

Les threads multiples par coeur peuvent aussi augmenter. L'idée est de profiter du memory stall pour faire progresser un autre thread pendant que la mémoire est accédée et les données rapatriées.

La manière la plus simple d'implémenter de l'hyperthreading est de dupliquer les registres de l'unité de calcul et d'alterner. Il y a ainsi deux pipelines, ce qui évite le stalling.
	
	\chapter{Gestion de la mémoire}

Avant qu'ils ne démarrent, les programmes doivent être rapatriés du disque vers la mémoire et placés dans un processus qui va l'exécuter. La mémoire principale ($\approx$ 100 cycles) et les registres (1 cycle) sont les seules unités de stockage que le CPU peut accéder directement. Afin de ne pas devoir effectuer de trop nombreux aller-retours entre les registres et la RAM, on place une cache entre les deux.

Il est nécessaire de protéger la mémoire des processus, c'est-à-dire qu'il faut empêcher qu'un programme ait écrire dans la mémoire d'autres programmes.

Chaque programme possède une adresse de base (qui aura pour adresse virtuelle 0) et une limite, qui sont stockées dans des registres et qui définissent l'espace d'adresses logiques. Vérifier une adresse logique doit être très rapide car elle est effectuée à chaque accès mémoire, c'est pour cela que c'est du hardware qui s'en charge.

\dessin{24}

\section{Liaison des instructions et des données à la mémoire}

L'address binding consiste à mapper une adresse d'un espace d'adresses à un autre. La liaison des instructions et des données à des adresses mémoires peut se produire à trois étapes différentes :

\begin{itemize}
	\item à la compilation (compile time) : si la localisation de la mémoire est connue d'avance, du code absolu peut être généré. Il faut recompiler le code si l'adresse de départ change.
	\item au chargement (load time) : il faut générer du relocatable code si la localisation mémoire n'est pas connue à la compilation. Par contre, une fois chargé, le programme doit rester en place.
	\item à l'exécution (execution time) : la liaison est reportée jusqu'à l'exécution si le processus peut être déplacé durant son exécution d'un segment de mémoire à un autre. Un support hardware est nécessaire pour le mapping des adresses (registres de base et de limite).
\end{itemize}

\dessinS{25}{0.55}

	\subsection{Adresses logiques et physiques}
	
	Les adresses logiques (ou virtuelles) sont générées par le CPU et sont utilisées pour séparer l'espace d'adresses physiques. Une adresse physique est l'adresse vue par l'unité de mémoire.
	
	Les adresses physiques et logiques sont ainsi les mêmes pour les schémas de liaison d'adresses à la compilation et au chargement, mais elles diffèrent à l'exécution.
	
	La traduction d'une adresse virtuelle en une adresse physique est effectuée par le MMU (Memory-Management Unit). La valeur du relocation register est ajoutée à chaque adresse générée par le processus utilisateur au moment où elle est envoyée en mémoire. Ainsi, l'utilisateur utilise des adresses logiques/virtuelles et n'est jamais confronté à des adresses physiques. Cela pourrait être le cas avec un schéma de liaison d'adresses à la compilation ou au chargement, mais il ne le saura pas.
	
	\dessinS{26}{0.35}
	
	\subsection{Chargement dynamique (dynamic loading)}
	
	Une routine n'est chargée que si elle est appelée, ce qui permet une meilleure utilisation de la mémoire vu qu'une routine non utilisée n'est jamais chargée. C'est aussi utile dans le cas où de larges portions de code ne sont utilisées qu'infréquemment. Il n'y a pas de support particulier nécessaire pour l'OS, ce comportement est implémenté à travers le design du programme.
	
	\subsection{Liaison dynamique (dynamic linking)}
	
	L'étape de linkage est postposée jusqu'à l'exécution du programme, où on ne charge que les librairies dont on a besoin.
	
	Une petite portion de code, le stub, est utilisée pour localiser la routine appropriée de la librairie en mémoire. Il se remplace alors par l'adresse de la routine et l'exécute, après que l'OS vérifié que la routine est dans l'espace d'adresses du processus.
	
	Le dynamic linking est particulièrement utile pour les librairies, qui peuvent être partagées entre plusieurs programmes (shared libraries). On peut également avoir différentes versions de la même librairie dynamique au même moment dans le système.

\section{Swapping}

Un processus peut être swappé temporairement hors de la mémoire dans une unité de stockage et rapatrié en mémoire pour continuer son exécution.

L'unité de stockage est généralement un disque suffisamment grand pour accueillir une copie de la mémoire de tous les utilisateurs. Elle doit également permettre un accès direct à des images de mémoire.

Roll out, roll in : variante du swap utilisée pour les algorithmes d'ordonnancement avec une priorité. Les processus de basse priorité sont swappés pour que les processus de plus haute priorité soient chargés et exécutés.

La plupart du temps de swapping est engendrée par le temps de transfert, qui est proportionnelle à la quantité de mémoire swappée.

Beaucoup de variantes du swapping peuvent être trouvées sur plusieurs systèmes différents. En général, le système maintient une liste similaire à une ready queue, mais qui contient des processus qui ont leur image mémoire sur le disque.

\begin{itemize}
	\item swap out (of memory) : action d'extraire un processus hors de la mémoire.
	\item swap in (memory) : action de récupérer un processus du disque et de le placer dans la mémoire.
\end{itemize}


\section{Allocation contiguë}

La mémoire principale est généralement partitionnée en deux :

\begin{itemize}
	\item la mémoire basse, qui comprend l'OS
	\item la mémoire haute, qui entrepose les processus utilisateurs
\end{itemize}

Des registres de relocalisation (relocation registers) sont utilisés pour protéger les processus utilisateurs les uns des autres et d'empêcher qu'il y ait des modifications dans le code et les données de l'OS :

\begin{itemize}
	\item base register : contient la valeur de la plus petite adresse physique
	\item limit register : contient l'intervalle d'adresses logiques ; chaque adresse virtuelle/logique doit être plus petite que ce registre
	\item le MMU mappe les adresses logiques dynamiquement.
\end{itemize}

\dessinS{27}{0.35}

Quand un processus arrive (de retour du disque ou nouvellement créé), il doit être placé en mémoire. L'OS va allouer de la mémoire située dans un "trou" (partition/bloc de mémoire disponible) suffisament large et l'y placer. L'OS stocke ainsi des informations sur les partitions allouées et celles qui sont libres.

Le problème est que les partitions libres sont éparpillées en mémoire, et qu'au final on peut arriver à des trous qui sont trop petits que pour pouvoir être utilisés. Une première mesure est de les fusionner quand ils sont adjacents.

\dessinS{28}{0.35}

	\subsection{Problème d'allocation d'espace dynamique}
	
	On cherche donc à satisfaire une requête de taille $n$ à partir d'une liste de partitions libres. Plusieurs stratégies existent :
	
	\begin{itemize}
		\item first-fit : on utilise la première partition qui convient
		\item best-fit : on utilise la plus petite partition qui est suffisamment grande, ce qui implique de parcourir entièrement la liste (sauf si elle est triée selon la taille des partitions). Cela produit des plus petites partitions ; le risque de fragmentation externe est plus élevé.
		
		\item worst-fit : on utilise la plus grande partition, ce qui implique aussi un parcours entier de la liste, mais produit des plus grandes partitions libres.
	\end{itemize}
	
	Les stratégies first-fit et best-fit sont meilleures que worst-fit en terme de vitesse et d'utilisation d'espace.
	
	\subsection{Fragmentation}
	
	Il existe deux types de fragmentation :
	
	\begin{itemize}
		\item la fragmentation externe : l'espace mémoire existe pour satisfaire une requête, mais n'est pas continu
		\item la fragmentation interne : la mémoire allouée peut être plus grande que la mémoire demandée. La différence de ces deux mémoires est interne à la mémoire du processus mais n'est pas utilisée.
	\end{itemize}
	
	On peut réduire la fragmentation externe avec une compaction : on rassemble toutes les partitions libres en un seul bloc. Cela est possible que si la relocation est dynamique et est faite à l'exécution.
	
	On a cependant un problème d'I/O : si un programme est déplacé, le buffer qui lui était attribué n'est plus utilisable car il est aussi déplacé et le périphérique I/O ne sera pas au courant. On a alors deux solutions :
	
	\begin{enumerate}
		\item ne pas bouger les programmes en attente d'I/O, mais cela réduit l'efficacité
		\item le device driver écrit dans la mémoire de l'OS, le noyau va alors transmettre les données à l'espace utilisateur. On a alors deux écritures pour toute opération I/O, elles sont plus lentes.
	\end{enumerate}
	
\section{Paging}

L'espace d'adresses physiques d'un processus ne peut pas être continu. On va alors 

\begin{itemize}
	\item diviser la mémoire physique en blocs de taille fixes, appelés frame (d'une taille de puissance 2, entre 512k et 8 192 bytes)
	\item diviser la mémoire logique en blocs de la même taille, appelés pages
\end{itemize}

On suit toutes les frames libres. Dès lors, lorsqu'on lance un programme de $n$ pages, il faut trouver $n$ frames libres et charger le programme. Chaque programme possède sa propre table des pages.

Les pages sont continues en mémoire virtuelle, mais par contre les frames qui leurs sont associées ne le sont pas. Il est alors nécessaire de mettre en place une table de pages pour traduire les adresses logiques en adresses physiques.

Le mécanisme de paging conduit nécessairement à de la fragmentation interne, mais permet de diminuer fortement la fragmentation externe.

	\subsection{Traduction d'adresses}
	
	Une adresse générée par le CPU est divisée en deux :
	
	\begin{itemize}
		\item le numéro de page (page number, $p$), qui est utilisé comme index dans la table des pages (contenant l'adresse de base de chaque page en mémoire physique)
		\item le décalage de page (page offset, $d$), qui est combiné à l'adresse de base pour définir l'adresse mémoire physique qui doit être envoyée au périphérique de stockage.
	\end{itemize}
	
	\dessin{29}
	
	On a alors des pages de taille $2^n$ et un espace mémoire de taille $2^m$. Des puissances de $2$ permettent de simplifier les opérations de numéro de page et d'offset.
	
	\dessinS{30}{0.5}
	
	Exemple d'allocation de pages :
	
	\dessin{31}
	
	\subsection{Implémentation de la table des pages}
	
	La page table est gardée en mémoire principale. On a deux registres :
	
	\begin{itemize}
		\item PTBR - Page-table base register : pointe la table
		\item PTLR - Page-table length register : indique la taille de la table
	\end{itemize}
	
	Avec ce schéma, chaque accès à une donnée ou une instruction requiert deux accès mémoire :
	
	\begin{itemize}
		\item un pour accéder à la table des pages
		\item un pour les données/l'instruction
	\end{itemize}
	
	Afin d'accélérer les opérations en diminuant les accès, on utilise un cache hardware à mémoire associative, appelé le TLB (translation look-aside buffer). Il y a un TLB par CPU. (TLB shootdown : communication entre des TLBs pour supprimer des entrées dans les autres TLBs.)
	
	Certains TLBs stockent dans chaque entrée un ASID (address-space identifier), qui identifie de manière unique chaque processus, afin d'offrir une protection de l'espace d'adresses pour ce processus et de permettre à plusieurs processus d'utiliser le TLB. S'il n'y a pas d'ASID, le TLB doit être vidé à chaque fois qu'une table de pages est sélectionnée, afin de ne pas délivrer des adresses erronées.
	
	Un TLB est une mémoire physique associative, où la recherche est effectuée en parallèle. Cela la rend plus rapide, mais consomme plus d'énergie et nécessite plus de circuits, du coup les comparateurs prennent plus de place, ce qui fait que le TLB est au final assez petit. Ainsi, si $p$ est dans ce registre associatif, on récupère le $d$ associé, sinon on va chercher $d$ dans la table des pages en mémoire.
	
	\dessinS{32}{0.5}
	
	Supposons qu'une boucle utilise 10 pages et que le TLB peut en contenir 100, donc 10\% de sa capacité est utilisée. Que faire si le processus est déscheduled ? Il va falloir, à chaque changement de contexte, invalider tous les TBL, donc les vider, mais il est possible que l'union des TLBs de plusieurs processus puisse tenir dans un TLB.
	
	
	\subsection{Temps d'accès effectif}
	
	Supposons que $\epsilon$ soit le temps nécessaire pour accéder au TLB et que chaque accès mémoire soit de 1 $\mu$s. Supposons également que le hit ratio (pourcentage de fois qu'un numéro de page est trouvé dans les TLBS (lié au nombre de registres)) est de $\alpha$. On a alors l'effective access time
	
	$$EAT = (1 + \epsilon) \alpha + (2 + \epsilon) (1 - \alpha)$$
	
	En cas de hit, on a donc $\epsilon +$ le temps d'un accès mémoire, tandis qu'un miss comporte $\epsilon$ et deux accès mémoires. Au final, on a
	
	$$EAT = 2 + \epsilon - \alpha$$
	
	\subsection{Protection de la mémoire}
	
	Un processus ne peut pas accéder à d'autres frames que les siennes, vu que ce sont les seules accessibles par sa table de pages.
	
	La protection de la mémoire est implémentée en associant un bit de protection à chaque frame. Ainsi, dans chaque entrée d'une page table, il y a un bit valid-invalid :
	
	\begin{itemize}
		\item "valid" indique que la page associée est dans l'espace logique du processus, et que donc c'est une page légale
		\item "invalid" indique que la page n'est pas dans l'espace logique du processus.
	\end{itemize}
	
	\dessin{33}
	
	\subsection{Pages partagées}
	
	Du code peut être partagé entre plusieurs processus. Dans ce cas, il est disponible en read-only et doit être réentrant et doit apparaître au même emplacement de la mémoire logique de tous les processus.
	
	Quand du code ou des données sont privées, chaque processus possède sa propre copie à part, et dans ce cas les pages de ce code et des données peuvent se trouver n'importe où dans l'espace logique.
	
	Après un fork, les pages du fils pointent vers les pages du parent et un bit est set. On ne recopie que les parties du processus parent qu'on modifie ; il y a une copie des frames sur demande.
	
	\dessin{34}
	
	\subsection{Structure d'une table de page}
	
	On a un problème avec les systèmes disposant de beaucoup de mémoire : vu qu'une frame représente une petite portion (par ex 4kB), il y aura énormément de pages à stocker (pour 4Go de ram, il y en aura 1 million).
	
	\subsubsection{Paging hiérarchique}
	
	Avec une hiérarchie, on divise l'espace d'adresses en plusieurs tables de pages. Une technique simple est une table de pages à deux niveaux.
	
	\dessinS{35}{0.5}
	
	Ainsi, l'adresse virtuelle possède plusieurs numéros de pages, qui vont permettre d'aller rechercher l'adresse physique en consultant plusieurs tables de pages.
	
	\dessinS{36}{0.5}
	
	Une telle technique peut s'utiliser sur autant de niveaux que l'on veut, mais elle a ses limites. En effet, plus il y a de niveaux et plus il faut effectuer de recherches. Or ces recherches sont faites à chaque utilisation de la mémoire.
	
	\subsubsection{Tables de pages hashée}
	
	C'est la technique généralement utilisée pour des espaces d'adresses de plus de 32 bits. Le numéro de page est hashé dans une table de pages, avec un chaînage en cas de collision.
	
	\dessinS{37}{0.5}
	
	\subsubsection{Tables de pages inversée}
	
	Il y a une entrée pour chaque page de mémoire réelle (physiquement allouée), qui consiste en l'adresse virtuelle de la page stockée dans ce vrai emplacement mémoire, avec des informations sur le processus qui possède la page.
	
	Cela permet de diminuer la mémoire nécessaire pour stocker chaque page table, mais cela augmente le temps nécessaire pour procéder à une recherche. On peut utiliser une hash table pour limiter les recherches à une ou quelques unes entrées de la table des pages. Un accès mémoire est nécessaire, deux si on utilise une table de hachage.
	
	Il n'y a donc qu'une seule page table dans le système, avec une taille adaptée, mais qui est plus lente. La table est dite inversée car elle est triée par rapport aux adresses physiques pour récupérer une adresser virtuelle.
	
	\dessinS{38}{0.5}
	
	Il y a des difficultés pour implémenter des pages partagées, car généralement ce sont plusieurs adresses virtuelles qui sont mappées à une même adresse physique, or il n'y a normalement qu'une entrée par frame dans cette table.
	
\section{Segmentation}

Il s'agit d'un schéma de gestion de la mémoire qui supporte une vue utilisateur de la mémoire. Un programme est vu comme un ensemble de segments, des unités logiques du programme (programme principale, procédure, objet, stack, etc).

\dessinS{39}{0.3}

Une adresse va alors consister en un tuple (numéro de segment, offset). On aura les structures de données suivantes :

\begin{itemize}
	\item une table de segments, qui permet à partir d'un numéro de segment d'obtenir une adresse physique à deux dimensions. Chaque entrée de cette table possède
	
	\begin{itemize}
		\item une base qui est l'adresse physique où le segment réside en mémoire
		\item une limite pour la longueur du segment
	\end{itemize}
	
	\item STBR - Segment-table base register : un registre qui pointe vers le début de la table en mémoire
	\item STLR - Segment-table length register : un registre qui indique le nombre de segments utilisés par un programme. De ce fait, un segment $s$ est légal si $s < STLR$.
\end{itemize}

Un mécanisme de protection existe : chaque entrée dans la table de segments est associée un bit de validation (s'il vaut 0, le segment est illégal) et à des privilèges d'écriture, lecture et exécution.

De la mémoire peut être partagée au niveau des segments. Vu que leur longueur est variable, l'allocation de mémoire pour les segments est un problème de stockage-allocation dynamique.

\dessinS{40}{0.5}

Si un segment est très gros, on peut utiliser un mécanisme de pages en dessous ; la table des segments associera un segment à une adresse virtuelle qu'il faudra convertir en adresse physique.

Désavantages de la segmentation :

\begin{itemize}
	\item il y a un risque de fragmentation externe
	\item lors de swap, il faut placer en mémoire des segments de taille variable
\end{itemize}
	
	\chapter{Gestion de la mémoire virtuelle}

La mémoire virtuelle sépare la mémoire logique utilisateur de la mémoire physique. Seules des parties du programme nécessitent d'être en mémoire lors de l'exécution, du coup un espace logique peut être beaucoup plus grand que l'espace physique.

La mémoire virtuelle permet également de partager des espaces de mémoire à plusieurs processus. Elle peut être implémentée via des demandes de pages ou de segments. Elle est généralement organisée comme ceci :

\dessin{41}

Une librairie partagée est généralement du code binaire qui n'est accessible qu'en lecture seule (cela permet de ne pas devoir la copier plusieurs fois en mémoire physique) par plusieurs processus. Cela n'a de toute façon pas de sens d'écrire dedans.

\dessinS{42}{0.3}


\section{Demande de page}

Une page n'est ramenée en mémoire que si elle est nécessaire. Cela permet d'avoir besoin de moins d'I/O et de mémoire, du coup la réponse est plus rapide. On permet également plus d'utilisateurs.

Cette politique a du sens car tout le code d'un programme n'est pas exécuté ; des pages peuvent être consommée mais jamais utilisée (par exemple les if des mallocs).

Lorsqu'une page est nécessaire, on regarde la référence. Si elle est invalide, on arrête l'opération, sinon on la ramène en mémoire.

Lazy swapper : on ne swap-in jamais une page tant qu'elle n'est pas demandée. Les swappers qui s'occupent de pages s'appellent des pagers.

Idéalement, les pages d'un même processus sont placées consécutivement sur le disque afin de pouvoir tout rapatrier plus rapidement (vu que le seek time est très grand sur un disque).

\dessinS{43}{0.45}

	\subsection{Bit de validité}
	
	Chaque entrée de la table des pages possèdent un bit valid-invalid qui indique si la page est légale et en mémoire (valid) ou non (invalid).
	
	Initialement, toutes les pages sont marquées comme invalides. Lors d'une traduction d'adresse, si le bit indique que la page est invalide, il y a un page fault.
		
	\subsection{Page fault}
	
	S'il y a une référence à une page, la première référence à cette page va déclencher un trap : un page fault. L'OS va alors regarder dans une autre table pour savoir si la référence est invalide (dans ce cas on annule l'opération et on renvoie une erreur), ou bien si elle n'est simplement pas en mémoire. Certains OS n'ont qu'un bit pour des raisons de compatibilité, lorsque le hardware ne sait stocker qu'un seul bit.
	
	Dans ce cas :
	
	\begin{itemize}
		\item on récupère une frame vide
		\item on swap la page dans la frame
		\item on réinitialise les tables
		\item on met le bit de validation à valid
		\item on recommence l'instruction qui a causé le page fault.
	\end{itemize}
	
	La réexécution de l'instruction est nécessaire dans certains cas :
	
	\begin{itemize}
		\item exécuter une instruction peut générer un page fault, car par exemple l'adresse dans le PC peut ne pas être en mémoire
		\item un page fault peut survenir sur les opérandes d'une opération
	\end{itemize}
	
	Cette réexécution est forcée parce que l'exécution d'une instruction n'est pas divisible (vu que le PC pointe l'instruction, et non le cycle dans lequel on se trouve).
	
	Par exemple, si on fait $\sharp 1 + \sharp 2 \rightarrow \sharp 3$ et si $\sharp 3$ fait un segfault, on devra recalculer $\sharp 1 + \sharp 2$. Si toutes les variables sont des pointeurs vers des valeurs dans des pages différentes, il peut y avoir au pire 7 page fault, donc il faut prévoir au moins 7 frames pour cette instruction (cela varie d'une instruction à l'autre). Si on ne peut pas avoir ce nombre minimum de pages en mémoire, l'instruction ne pourra jamais être exécutée. De plus, les blocs qui n'ont pas généré de page fault peuvent être déplacés pendant le traitement d'un autre page fault.
	
	$$\underbrace{\text{add}}_{1\text{ page}} \underbrace{\$1, \$2, \$3}_{\substack{3 \times 1 \text{ page si direct addressing}\\3 \times 2 \text{ si indirect addressing (pointeurs)}}}$$
	
	\note{Il faudrait parler ici du transparent 9.13 (rien compris).}
	
	\dessinS{44}{0.5}
	
	Supposons que le taux de page fault est $0 \leq p \leq 1$ (si $p = 0$, aucun page faults ; si $p = 1$, toutes les références produisent des page faults). On a l'EAT (effective access time) :
	
	$$EAT = (1 - p) \times \text{accès mémoire } + p (\text{ page fault overhead } + \text{ swap page out } + \text{ swap page in } + \text{ restart overhead})$$
	
	Afin de réduire les accès disque, on peut compresser les pages et les garder en mémoire.
	
\section{Création de processus}

La mémoire virtuelle est bénéfique dans la création de processus, avec le copy-on-write et les memory-mapped files.

	\subsection{Copy-on-Write}
	
	La copy-on-write (COW) permet au parent et à son enfant de partager une page en mémoire tant qu'elle n'est pas modifiée. Si elle l'est, alors seulement à ce moment-là la page est copiée. Cela rend plus efficace la création de processus vu que seules les pages modifiées sont copiées.
	
	Les pages libres sont situées dans une pool de pages ne contenant que des 0. Si jamais il n'y a plus de page libre, on cherche une page en mémoire qui n'est pas trop utilisée et on la swap sur le disque. Cela demande l'utilisation d'un algorithme qui minimise le nombre de page faults. De plus, une même page peut être rapatriée en mémoire plusieurs fois.
	
	
\section{Page Replacement}
	
	On veut toutefois prévenir la sur-allocation de mémoire en modifiant le service de page fault pour inclure du page replacement. Ce dernier inclut un bit de modification (dirty bit, modifié par le hardware) qui permet de réduire l'overhead des transferts de page : seules les pages modifiées sont écrites sur le disque. C'est le page replacement qui permet de bien séparer les mémoires logique et physique, et qui du coup permet d'utiliser une mémoire virtuelle plus large que la mémoire physique.
	
	Un remplacement de page simple s'intègre comme ceci :
	
	\begin{enumerate}
		\item on cherche la page désirée sur le disque
		\item on cherche une frame vide
		
		\begin{itemize}
			\item si on trouve une frame vide, on l'utilise
			\item si on ne trouve pas de frame, on utilise l'algorithme de page replacement pour trouver une frame victime.
		\end{itemize}
		
		\item un rapatrie la page désirée dans la frame (nouvellement) libre et on met à jour la page et la table des pages
		\item on reprend le processus
	\end{enumerate}
	
	\dessinS{45}{0.45}
	
	
	\subsection{Algorithme de remplacement}
	
	On désire le plus petit taux de page fault.
	
		\subsubsection{FIFO}
		
		\dessinS{46}{0.35}
		
		Anomalie de Belady : plus il y a de frames, plus il y a de page faults.
		
		\dessinS{47}{0.3}
		
		\subsubsection{Algorithme optimal}
		
		On remplace la page qui ne sera pas utilisée pour la plus longue période. C'est un algorithme idéal, en pratique il n'est pas possible de l'implémenter, mais il permet de comparer les performances d'autres algorithmes.
		
		\dessinS{48}{0.35}
		
		\subsubsection{Algorithme LRU - Least Recently Used}
		
		\dessinS{49}{0.35}
		
		Pour implémenter cet algorithme, on assigne à chaque page un compteur. Chaque fois qu'une page est référencée, on copie la clock dans le compteur. Quand une page doit être changée, on regarde ces compteurs pour choisir celle que l'on va swapper.
		
		On pourrait simplifier la structure en utilisant une pile implémentée par une liste doublement liée. Lorsqu'une page est référencée, il suffit de la déplacer sur le sommet (nécessite 6 changements de pointeurs). La recherche pour le remplacement est ainsi extrêmement réduite.
		
		
		\subsubsection{Algorithme LRU (approximé)}
		
		On va assigner à chaque page un bit de référence, qui vaut initialement 0. Quand la page est référencée, on le met à 1. L'algorithme va alors remplacer une page dont le bit de référence vaut 0 (si elle existe). Le problème est qu'on n'a pas de notion d'ordre.
		
		Dès lors, on va utiliser le sens horaire : si une page qui va être remplacée (ordre horaire) a un bit de référence à 1, alors on met le bit de référence à 0, on laisse la page en mémoire et on remplace la prochaine page de la même façon.
		
		\dessinS{50}{0.5}
		
		
		\subsubsection{Algorithmes de comptage}
		
		On garde un compteur sur le nombre de références qui ont été faites à chaque page. Ainsi,
		
		\begin{itemize}
			\item un algorithme LFU va remplacer la page avec le plus petit compteur
			\item un algorithme MFU va remplacer la page avec le plus grand compteur ; il suppose qu'une page avec un petit compteur a été probablement ramenée en mémoire et doit encore être utilisée.
		\end{itemize}
		
		Le problème avec la technique du LFU est qu'une page utilisée pendant une initialisation peut avoir un grand compteur, mais ne sera pas réutilisée par après. D'un autre côté, une page fraîchement rapatriée aura un petit compteur, mais elle sera plus prompte à être swappée que l'autre page.
		
\section{Allocation de frame}

Chaque processus a besoin d'un minimum de pages pour pouvoir fonctionner.

	\subsection{Types d'allocation}
		\subsubsection{Allocation fixe}
	
			\paragraph{Allocation égale}
		
			Chaque processus reçoit un nombre égal de frame ; s'il y a 5 processus et 100 frames disponibles, chacun en recevra 20.
		
			\paragraph{Allocation proportionnelle}
		
			L'allocation dépend de la taille des processus. Soit $s_i$ la taille du processus $p_i$, $S = \sum_i s_i$ et $m$ le nombre total de frame. L'allocation $a_i$ pour un processus $p_i$ est
		
			$$a_i = \frac{s_i}{S}m$$
				
		\subsubsection{Allocation avec priorités}
	
	On utilise une allocation proportionnelle, mais qui utilise des priorités au lieu de la taille. Ainsi, si un processus $p_i$ génère une faute, on sélectionne une de ses frames pour un remplacement, tandis qu'on sélectionne une autre frame d'un processus avec une plus petite priorité.
	
	Donner plus de frames à un processus est une forme de priorité, car les processus qui en auront peu auront plus de page fault.
	
	\subsection{Allocations globale et locale}
	
	\begin{itemize}
		\item Remplacement global : un processus sélectionne une frame parmi toutes celles disponibles, ce qui inclut les frames d'autres processus. C'est une technique plus flexible, mais qui conduit à plus de trashing.
		
		\begin{itemize}
			\item[+] efficace ; il y a un meilleur throughput
			\item[-] un processus ne peut contrôler son taux de page fault
		\end{itemize}
		
		\item Remplacement local : chaque processus ne peut choisir que parmi ses propres frames allouées.
		
		
		\begin{itemize}
			\item[-] cela peut brider un processus en ne lui permettant pas d'aller chercher ailleurs des frames moins utilisées.
		\end{itemize}
	\end{itemize}
	
	
	\subsection{Trashing}
	
	Le trashing est le fait qu'un processus soit occupé à swapper des pages in et out. Cela peut arriver s'il ne possède pas assez de pages, et donc si le taux de page-fault est élevé (si le processus possède peu de pages qui sont toutes actives). Cela conduit à une faible utilisation du CPU, ce qui fait que l'OS croit qu'il est nécessaire d'augmenter le degré de multiprogramming, et donc d'ajouter un nouveau processus au système. Un nouveau processus aura besoin de nouvelles pages pour s'exécuter, ce qui provoquera encore plus de trashing.
	
	\dessinS{51}{0.3}
	
	On peut limiter le trashing avec un algorithme de replacement local, car il n'y a pas de vol de frames à d'autres processus.
	
	Le demand paging fonctionne grâce au modèle de localité : les processus migrent d'une localité à une autre, or celles-ci peuvent se recouvrir.
	
	Le trashing se produit quand la somme totale des localité est plus grande que l'entièreté de la mémoire.
	
	Le trashing peut être évité si chaque processus a assez de pages ; ce nombre est estimé par une stratégie comme le working-set model ou en gardant un taux de page fault compris entre deux bornes.
	
	\subsection{Working-Set Model}
	
	Soit $\Delta$ la fenêtre de working-set, qui est un nombre fixe de références de page (ex : 10 000 instructions).
	
	On définit $WSS_i$ (working set of process $p_i$) comme étant le nombre total de page référencées dans le plus récent $\Delta$ :
	
	\begin{itemize}
		\item si $\Delta$ est trop petit, il n'englobera pas une localité entière. S'il y a trop peu de frames pour un processus, c'est mauvais pour tous les autres car cela va provoquer du trashing, donc des accès disque, et donc les autres page fault seront ralentis.
		\item si $\Delta$ est trop grand, il englobera plusieurs localités. Trop de frames pour un processus est mauvais pour les autres car ils n'en ont pas beaucoup à se partager.
		\item si $\Delta = \infty$, il englobera le programme entier.
	\end{itemize}
	
	Soit $D = \sum_i WSS_i =$ la demande totale de frames. Si $D > m$, on a du trashing. Dans ce cas, une mesure serait de suspendre un des processus.
	
	\dessinS{52}{0.35}
	
	Pour garder un suivi du working set d'un processus, on peut utiliser un timer à intervalle avec un bit de référence.
	
	
	\dessin{53}
	
	
	\dessin{55}
	
	\subsection{Contrôle du taux de page fault}
	
	Il faut donc définir un taux de page fault acceptable. Il ne doit pas être trop bas sinon les processus perdent des frames, ni trop haut sinon certains en gagnent.
	
	\dessin{54}
	
	
\section{Memory-Mapped files}

Un memory-mapped file permet de traiter les files I/O avec des routines d'accès mémoire, en mappant un bloc du disque à une page en mémoire.

Un fichier est initialement lu en utilisant du demand paging, ce qui fera qu'une portion de la taille d'une page du fichier est lu du système de fichiers dans une page physique. Les lectures et écritures du fichier seront alors traitées comme des accès en mémoire classiques. Il y a donc une simplification du traitement de fichiers en passant par la mémoire plutôt qu'avec des appels systèmes write() et read().

Cela permet aussi à plusieurs processus de mapper le même fichier, et donc de partager des pages en mémoire.

\dessinS{56}{0.45}

\section{Allocation de mémoire kernel}

La mémoire du noyau est traitée différemment de la mémoire utilisateur, car

\begin{itemize}
	\item il y a des réservation de petites structures, or il est nécessaire de minimiser la consommation de l'OS.
	\item il est nécessaire que la mémoire physique soit consécutive pour faciliter/permettre des opérations en hardware.
\end{itemize}

Elle est généralement allouée à partir d'une pool de mémoire libre. Les requêtes de mémoire kernel portent sur des structures de tailles variées ; parfois, la mémoire réservée doit être continue.

\subsection{Buddy System}

Ce type d'allocation permet d'allouer de la mémoire d'un segment de taille fixe constitué de pages physiquement continues. L'allocateur utilisé est basé sur des puissances de 2 : les requêtes sont arrondies à la prochaine puissance de 2 (si elle n'en sont pas une), et lorsqu'une plus petite allocation est nécessaire que ce qui est disponible, on divise le segment courant en deux "buddies" de puissance 2 inférieure. On continue jusqu'à avoir la taille de segment appropriée.

\dessinS{57}{0.3}

Si on désalloue $C_L$, il y a une reformation du segment. Ce qui n'est pas utilisé ($C_R$, $B_R$, etc) est utilisé pour autre chose.

L'utilisation de grosses pages a des avantages et inconvénients :
\begin{itemize}
	\item[+] le seektime est plus rentabilisé, grâce à la localité des pages
	\item[+] les pages peuvent être de taille plus petite
	\item[-] si on utilise peu de mémoire, on a une grosse réservation (64k pour 33K par exemple)
	\item[-] on peut arriver à conserver des données inutiles
\end{itemize}

\subsection{Slab Allocator}

On définit un slab comme une ou plusieurs pages physiquement consécutives. Un cache consiste en un ou plusieurs slabs.

\dessin{58}

Un cache est dédié à chaque type d'objets du noyau ; chaque cache est rempli d'instances de ces structures de données. Lorsque le cache est créé, tous ces objets sont marqués comme libres. Quand des structures sont stockées, les objets sont marqués comme utilisés.

Si un slab est rempli d'objets utilisés, le prochain objet est alloué à partir d'un slab vide. S'il n'y a pas de slab vide, un nouveau slab est alloué.

\begin{itemize}
	\item[+] pas de fragmentation
	\item[+] les requêtes en mémoire sont satisfaites rapidement
\end{itemize}

\section{Autres problèmes}

	\subsection{Prepaging}
	
	Le prepaging a pour but de réduire les grands nombres de page faults qui se produisent au lancement d'un programme. Pour ce faire, on prépare toutes ou une partie des pages qui seront nécessaires au processus, avant même qu'elles soient référencées.
	
	Le problème est que ces pages prepaged sont inutilisées, et que donc de l'I/O et de la mémoire sont gaspillés. Supposons que l'on ait $s$ pages prepaged et $\alpha$ de ces pages sont utilisées. Est-ce que le coût de $s \times \alpha$ page faults évités est meilleur que le coût de prépager $s \times (1 - \alpha)$ pages non nécessaires ? Si $\alpha$ est proche de 0, alors le prepaging est perdant.
	
	
	\subsection{Taille de la page}
	
	Le choix de la taille de la page a un impact sur beaucoup de choses :
	
	\begin{itemize}
		\item la fragmentation : plus les pages sont petites et moins il y a de fragmentation
		\item la taille de la table des pages : plus les pages sont petites et plus cette table sera grande (car il y a plus de pages à stocker en mémoire)
		\item l'overhead I/O
		\item la localité : plus les pages sont grandes et plus on a de chance que la prochaine donnée/instruction référencée sera dans les pages qui sont déjà présentes en mémoire. Plus les pages sont petites et plus on peut suivre la localité (better tracking of locality)
	\end{itemize}
	
	
	\subsection{Buffers pour des I/O}
	
	Il y a des problèmes avec les opérations I/O qui ont des buffers en mémoire : les frames peuvent être changées alors que le périphérique I/O n'est pas au courant. Il y a deux solutions :
	
	\begin{itemize}
		\item mettre les buffers dans le noyau (mais cela implique des copies de l'espace kernel vers l'espace utilisateur en plus)
		\item locker la page
	\end{itemize}
	
	\subsection{TLB Reach}
	
	La taille du TLB est réduire ; l'augmenter coûte cher vu le hardware nécessaire.
	
	\subsection{Structure du programme}
	
	La structure du programme peut influencer grandement les performances, par exemple \texttt{for(j) for(i)} à la place de \texttt{for(i) for(j)}.
	
	\chapter{Systèmes de fichiers}

Un fichier est un espace continu de données que l'on peut adresser.  Un fichier peut contenir des données  (numériques, caractères, binaires) ou un programme, un binaire mais avec une structure particulière qui permet de l'exécuter via l'OS.

L'OS est chargé de transformer cette vision de haut niveau en blocs sur le disque.

\section{Structure des fichiers}

La structure peut être

\begin{itemize}
	\item inexistante, avec une simple liste de bytes
	\item simple : lignes, longueur fixée ou variable
	\item complexe : document formaté ou relocatable load file
\end{itemize}

Les fichiers simples et complexes peuvent être simulés par des simples séquences de bytes en insérant des caractères de contrôle appropriés.

Ce sont les programmes et l'OS qui créent les fichiers.

Si un OS pouvait comprendre tous les types de fichiers :

\begin{itemize}
	\item[+] plus facile de programmer une application
	\item[-] il y a beaucoup trop de fichiers différents, cela donnerait lieu à trop de code
	\item[-] un programmeur ne peut pas créer de nouveaux types de fichiers ; il est bloqué à ce que comprend l'OS
\end{itemize}


\section{Attributs d'un fichier}

\begin{itemize}
	\item un nom (la seule information permettant une lecture par un humain)
	\item un identifiant unique, qui identifie le fichier dans le système de fichiers
	\item un type, pour les systèmes qui supportent différents types de fichier
	\item un emplacement, un pointeur vers la position du fichier sur un device
	\item une taille
	\item une protection, qui contrôle qui peut lire, écrire et exécuter le fichier
	\item un temps, une date et une identification d'utilisateur, qui servent à la protection, à la sécurisation et au monitoring de l'utilisation du fichier
\end{itemize}

Ces informations sont gardées dans la structure de répertoire (directory structure), qui est maintenue sur le disque.

\section{Opérations sur les fichiers}

Un fichier est un type de données abstraites (abstract data type), c'est-à-dire que c'est un type de données avec des opérations permettant de les manipuler. Ce n'est donc pas une structure de données (data structure), qui est plutôt utilisée pour représenter les données. Vu que le type est abstrait, cela veut dire qu'il peut y avoir plusieurs instances dans un programme.

On a les opérations suivantes :

\begin{itemize}
	\item create, write, read
	\item reposition within file ; moyen de savoir à quelle position on se trouve dans le fichier lorsqu'on lit ou écrit. La position est prise par rapport au début ou à la fin du fichier.
	\item delete
	\item truncate (par rapport à delete, cela ne supprime pas les attributs du fichier ; ça ne fait que le vider)
	\item Open($F_i$) : cherche l'entrée $F_i$ dans la structure de répertoire sur le disque et déplace le contenu en mémoire.
	
	\item Close($F_i$) : déplace le contenu en mémoire de $F_i$ vers la structure de répertoire sur le disque.
\end{itemize}

Vu que le directory structure est sur le disque, un accès systématique et répété serait trop lent. Dès lors, lorsqu'un fichier est ouvert, l'OS va charger les méta-données/attributs du fichier en mémoire et le retirer lors de la fermeture.

Un disque ne comprend que des blocs ; l'OS bufferise des blocs lors de la lecture ou l'écriture de fichier. Du coup, ne pas fermer un fichier lors d'une lecture n'est pas grave, pas contre ne pas fermer un fichier après une écriture peut entraîner la non-écriture des données du buffer sur le disque, donc la perte de données.

Plusieurs fichiers pourraient accéder au fichier en même temps. Pour gérer cela, il faut garder un pointeur vers chaque processus dans une table, avec des méta-données.

Si jamais il y a une demande de lecture et une demande de truncate en même temps, on va dupliquer le fichier avant de le vider.

	\subsection{Ouverture de fichier}
	
	Plusieurs structures de données sont nécessaires pour gérer l'ouverture de fichiers
	
	\begin{itemize}
		
		\item file table : table commune et qui contient les méta-données de tous les fichiers ouverts
		\item Chaque processus a une table qui liste les fichiers utilisés, avec la position, un pointeur vers les méta-données et les droits d'accès
		\item file-open count : compteur du nombre de fois que le fichier est ouvert, afin de pouvoir retirer des données de la table des fichiers ouverts lorsque les processus le ferment
	\end{itemize}
	
	\subsection{Ouverture de fichier avec verrou}
	
	Cette opération est proposée par les OS et les systèmes de fichier et permet de négocier l'accès à un fichier.
	
	Il y a deux modes d'utilisation :
	
	\begin{itemize}
		\item mandatory : l'accès est refusé selon le type de verrou tenu et demandé. Ainsi, openlock() et read() sont autorisés par d'autres processus, mais pas write().
		\item[$\rightarrow$] le verrou est imposé à l'ouverture ; un seul processus a le verrou, l'OS refusera de l'attribuer à un autre
		\item advisory : les processus peuvent savoir le statut des verrous et décident de ce qu'ils vont faire ; c'est au programmeur de tout gérer.
		\item[$\rightarrow$] il faut utiliser openlock() pour locker proprement le fichier. C'est le software qui gère le lock, l'OS n'empêchera pas l'acquisition du verrou.
	\end{itemize}
	
	
\section{Types de fichiers}

\dessin{59}

Sous Unix, les extensions sont généralement ignorées et des magic numbers sont utilisés pour représenter les types de fichier.

%%%%%%%%%%%%%%%%%%%%%%%%%%%%%%%%%%%%%%%%%%
%%%% Suite à compléter avec les notes %%%%
%%%%%%%%%%%%%%%%%%%%%%%%%%%%%%%%%%%%%%%%%%

\section{Méthodes d'accès}

Deux accès sont possibles :

\begin{itemize}
	\item accès séquentiel, avec
	
	\begin{itemize}
		\item read next
		\item write next
		\item reset
		\item no read after last write
		\item rewrite
	\end{itemize}
	
	\dessinS{60}{0.3}
	
	\item accès direct, avec (si $n$ est un numéro de bloc relatif)
	\begin{itemize}
		\item read n
		\item write n
		\item position to n
		\item read next, write next
		\item rewrite n
	\end{itemize}
\end{itemize}

Un accès séquentiel peut être simulé par un accès direct :

\begin{itemize}
	\item reset :
	\begin{itemize}
		\item cp = 0;
	\end{itemize}
	
	\item read next
	\begin{itemize}
		\item read cp;
		\item cp = cp + 1;
	\end{itemize}
	
	\item write next
	\begin{itemize}
		\item write cp;
		\item cp = cp + 1;
	\end{itemize}
\end{itemize}

Afin d'indexer un fichier (de manière relative), on maintient un fichier d'index. (?)

\dessin{61}

\section{Directory Structure}

Un directory structure est une connexion de noeuds qui contiennent des informations à propos de tous les fichiers. Le directory structure et les fichiers résident tous sur le disque.

	\subsection{Structure du disque}
	
	Un disque peut être divisé en partitions. Du RAID permet de protéger des disques ou des partitions d'une défaillance.
	
	Un disque ou une partition peut être utilisée brute (raw), c'est-à-dire sans système de fichiers, ou formatée avec un système de fichiers.
	
	Les partitions sont également connues sous le nom de minidisques ou de tranches. L'entité qui contient le système de fichiers est connu sous le nom de volume.
	
	Chaque volume contenant un système de fichiers garde des informations dessus dans un device directory ou une volume table of contents.
	
	Des systèmes de fichiers généraux (general-purpose file systems) existent tout comme des systèmes de fichiers spéciaux (special-purpose file systems) (par exemple le swap). Généralement ils sont tous au sein du même OS ou ordinateur.
	
	\dessin{62}
	
	\subsection{Opérations sur un répertoire}
	
	\begin{itemize}
		\item chercher un fichier
		\item créer, supprimer, renommer un fichier
		\item lister un répertoire
		\item parcourir le système de fichiers
	\end{itemize}
	
	\subsection{Organisation logique d'un répertoire}
	
	Le but est d'obtenir
	
	\begin{itemize}
		\item de l'efficacité, notamment localiser un fichier rapidement
		\item la possibilité de nommer les fichiers, ce qui est plus simple pour les utilisateurs. Ainsi, deux utilisateurs peuvent avoir le même nom pour différents fichiers, et le même fichier peut avoir plusieurs noms différents
		\item la possibilité de grouper des fichiers par leurs propriétés (tous les programmes, tous les jeux, etc)
	\end{itemize}
	
		\subsubsection{Répertoire à un niveau}
		
		On n'a qu'un seul répertoire pour tous les utilisateurs. Un fichier est associé directement à lui-même.
		
		\dessinS{63}{0.3}
		
		\begin{itemize}
			\item[+] facile à maintenir (par exemple, une recherche binaire est possible après avoir trié les noms) 
			\item[-] groupage impossible
			\item[-] les noms doivent être uniques
		\end{itemize}
		
		
		\subsubsection{Répertoire à deux niveaux}
		
		Chaque utilisateur possède un répertoire séparé. Un chemin (path name) est alors utilisé pour distinguer un fichier.
		
		\dessinS{64}{0.35}
		
		\begin{itemize}
			\item[+] recherche relativement efficace
			\item[+] plusieurs fichiers peuvent avoir le même nom,... 
			\item[-] ... mais uniquement pour des utilisateurs distincts
			\item[-] pas de capacité de groupage
			\item[-] un utilisateur ne peut pas avoir deux fichiers qui portent le même nom
		\end{itemize}
		
				
		\subsubsection{Répertoire avec une structure d'arbre}
		
		\dessin{65}
		
		\begin{itemize}
			\item[+] recherche efficace
			\item[+] capacités de groupage
		\end{itemize}
		
		On a la possibilité d'utiliser des chemins relatifs ou absolus.
		
		On a le choix entre deux polices pour supprimer un répertoire :
		
		\begin{itemize}
			\item supprimer récursivement tout son contenu
			\item ne rien supprimer tant que le répertoire contient des fichiers
		\end{itemize}
		
		\subsubsection{Répertoire à graphe acyclique}
		
		On a le partage de sous-répertoires et de fichiers.
		
		\dessin{66}
		
		On peut avoir deux noms différents pour le même fichier : ce sont des alias.
		
		Si par exemple dict supprime list, on a un pointeur vacant. Plusieurs solutions sont possibles :
		
		\begin{itemize}
			\item utiliser des backpointers, afin que l'on puisse supprimer tous les pointeurs. On a cependant des enregistrements avec une taille variable.
			\item utiliser des backpointers utilisant une organisation de daisy chain.
			\item utiliser une solution avec un compteur entry-hold.
		\end{itemize}
		
		On a un nouveau type d'entrée : un lien, qui pointe vers un fichier existant. La résolution du lien consiste à suivre le pointeur.
		
		\subsubsection{Répertoire à graphe général}
		
		\dessin{67}
		
		Comment garantir qu'il n'y a pas de cycle :
		
		\begin{itemize}
			\item ne permettre que des liens vers des fichiers, et pas vers des sous-répertoire
			\item garbage collection
			\item chaque fois qu'un nouveau lien est ajouté, on utilise un algorithme de détection de cycle pour voir si c'est bon.
		\end{itemize}
		
		Si on fait ls avi, on va avoir une boucle, ce qu'on veut éviter. Pour parer à cette situation, on va utiliser un bit qui indique s'il s'agit d'un lien vers quelque chose qui existe déjà ou si c'est un lien direct vers un répertoire/fichier créé. De ce fait, on ne suit que les liens directs.
	
		Si on supprime book, on peut effectuer un comptage, et s'il y a plus d'un direct link, on ne supprime rien. Un lien sera supprimé uniquement quand il est accédé.
		
		Le garbage collector n'est pas une bonne solution car il peut devenir énorme et résider sur le disque (donc, lorsqu'on en aura besoin, il y aura du swap et donc un temps d'attente important).
		
		La structure peut être importante et résider sur le disque. Lazzy link collection : on collecte les liens "morts" lorsqu'ils sont utilisés.
		
		\dessinS{68}{0.3}
		
\section{Montage de systèmes de fichiers}

Un système de fichiers doit être monté avant qu'il ne soit accédé, c'est-à-dire que l'on rend compte à l'OS qu'on veut pouvoir l'utiliser. Un systèmes de fichiers non monté est monté à un point de montage (mount point).

\dessin{69}

Lorsqu'on monte les points non montées (dans users), on a trois possibilités :

\begin{itemize}
	\item refuser le montage
	\item fusionner les sous-répertoires
	\item masquer les sous-répertoires déjà présents
\end{itemize}

\section{Partage de fichiers}

Le partage de fichiers sur un systèmes à plusieurs utilisateurs est généralement souhaité, mais cela doit se faire avec un mécanisme de protection. Sur des systèmes distribués, les fichiers partagés peuvent l'être à travers un réseau. 

La méthode NFS (Network File System) est une méthode de partage de fichiers distribués assez courante.

Pour permettre le partage de fichiers, on utilise des identifiants utilisateurs (users IDs), qui permettent d'attribuer des permissions et des protections par utilisateur, et des identifiants de groupe (group IDs), qui donnent aux utilisateurs du groupe les permissions du groupe.

	\subsection{Systèmes de fichiers distants}
	
	On utilise le réseau pour permettre à un système de fichiers d'être accessible par plusieurs systèmes. Cela peut se faire
	
	\begin{itemize}
		\item manuellement, avec des programmes comme FTP
		\item automatiquement, de manière transparente, en utilisant des systèmes de fichiers distribués
		\item semi-automatiquement, via le world wide web
	\end{itemize}
	
	Le modèle client-serveur permet aux clients de monter des systèmes fichiers distants de serveurs. Un serveur peut avoir plusieurs clients, mais le système d'identification reste insécurisé ou compliqué.
	
	NFS est le protocole de partage de fichiers standard UNIX, tandis que CIFS est le protocole standard de Windows. Des services distribués d'information (distributed naming services) comme LDAP, DNS, NIS, etc donnent les informations nécessaires pour faire du remote computing.
	
	A cause de l'aspect distant, il est nécessaire de prendre en compte de nouveaux modes de défaillance.  La récupération d'une défaillance peut impliquer les informations sur chaque requête distance. Ainsi, des protocoles stateless comme NFS incluent toutes ces informations dans chaque requête, ce qui permet une récupération facile mais est moins sécurisé.
	
	
\section{Protection}
	
Le créateur/propriétaire d'un fichier devrait être capable de contrôler ce qui peut être fait avec et par qui. Ainsi, on a les types d'accès suivants :
	
	\begin{itemize}
		\item read
		\item write
		\item execute
		\item append
		\item delete
		\item list
	\end{itemize}
	
	Trois modes d'accès sont disponibles : read, write et execute. Il y a trois classes d'utilisateurs : l'accès propriétaire, l'accès groupe et l'accès public.
	
	Attacher un fichier à groupe : chgrp G file.
	
	\chapter{Implémentation d'un système de fichiers}

Une structure de fichier est une unité logique de stockage à laquelle sont attachées des informations.

Un système de fichiers est organisé en couches et réside sur des périphériques de stockage secondaire (disques). Il fournit un moyen d'accès convénient et efficace, et permet de stocker, localiser et récupérer des données facilement.

Un file control block est une structure de stockage qui est constituée d'informations à propos d'un fichier.

Le device driver contrôle le périphérique physique.

\dessin{70}

Le module d'organisation des fichiers permet la traduction des blocs logiques de l'OS en blocs physiques.

Un boot control block contient les informations nécessaires au système pour booter l'OS à partir d'un volume.

Le volume control block contient des détails sur un volume.

Les structures de répertoire organisent les fichiers.

Chaque détail d'un fichier est stocké dans un File Control Block (FCB). Il contient généralement

\begin{itemize}
	\item les permissions du fichier
	\item des dates (création, accès, écriture)
	\item le propriétaire, le groupe, l'ACL (access control list)
	\item la taille du fichier
	\item les blocs de données du fichier, ou un pointeur vers ces blocs
\end{itemize}

\section{Structures en mémoire}

Structures en mémoire lors de (a) l'ouverture d'un fichier et (b) d'une lecture.

\dessinS{71}{0.45}

\section{Systèmes de fichiers virtuels}

Les systèmes de fichiers virtuels (VFS) offrent une approche orientée objet pour implémenter les systèmes de fichiers. Cela permet au système d'appeler une même interface (l'API) pour plusieurs types de systèmes de fichier. Cette API est préférée plutôt qu'une interface spécifique à chaque système de fichiers.
\dessinS{72}{0.45}

\section{Implémentation d'un répertoire}

Deux structures sont généralement utilisées :

\begin{itemize}
	\item une liste liée des noms de fichiers avec un pointeur vers les données. C'est simple à programmer, mais l'exécution est plus lente.
	\item une table de hachage, une liste liée avec une structure de hash. Cela diminue le temps de recherche, mais on est face à des collisions lorsque deux noms de fichier se hashent au même emplacement. De plus, la taille de la table est fixe.
\end{itemize}


\section{Méthodes d'allocation}

	Ces méthodes décrivent comment les blocs du disques sont alloués pour des fichiers.

	\subsection{Allocation contigüe}
	
	Chaque fichier occupe un ensemble continu de blocs sur le disque.
	
	\begin{itemize}
		\item[+] c'est simple ; on n'a besoin que d'une position de départ (un numéro de bloc) et de la longueur
		\item[+] un accès aléatoire est possible
		\item[+] la lecture du fichier est rapide : on n'a qu'un seek time et un read time suivi de plus petits read times.
		\item[-] gaspillage de place car des blocs ne sont pas utilisés. On revient à un problème d'allocation dynamique à cause des blocs libres éparpillés partout (fragmentation externe, entre les fichiers)
		\item[-] les fichiers ne peuvent pas s'agrandir
	\end{itemize}
			
	\subsection{Extent-Based systems}
	
	Il s'agit d'un système où les blocs du disque sont alloués par extents. Un extent est un ensemble de blocs continus ; un fichier consiste en un ou plusieurs extends.
	
	Cela permet de diminuer la fragmentation externe, mais augmente la fragmentation interne.
	
	Par exemple, si un extent fait 10 blocs et qu'on a un fichier de 27 blocs, il y a 3 blocs qui sont gaspillés.
	
	\subsection{Allocation liée}
	
	Chaque fichier est une liste liée de blocs disque. Ceux-ci peuvent être éparpillés un peu partout sur le disque.
	
	\dessin{73}
	
	\begin{itemize}
		\item[+] c'est simple à programmer ; on n'a besoin que de l'adresse de départ
		\item[+] il n'y a pas de gaspillage d'espace ; pas de fragmentation (sauf de la fragmentation interne, dans le bloc)
		\item[+] le fichier peut grandir
		\item[-] plus lent, car il faut accéder à plusieurs endroits sur le disque
		\item[-] pas de random access
		\item[-] la liste peut être brisée par un bloc défectueux
	\end{itemize}
	
	La FAT (file-allocation table) est utilisée par MS-DOS et contient les pointeurs vers les blocs suivants, donc permet un random access. Elle contient une entrée par bloc et permet de suivre les blocs d'un fichier. Elle est stockée sur le disque et gardée en mémoire.
	
	\dessin{74}
	
	\subsection{Allocation indexée}
	
	Tous les pointeurs vers les blocs sont rassemblés dans un bloc d'index.
	
	\dessin{75}
	
	\begin{itemize}
		\item[+] random access
		\item[+] accès dynamique sans fragmentation externe...
		\item ... mais overhead à cause du bloc d'index.
		\item[-] nécessité d'avoir une table d'index
		\item[-] perdre le bloc index fait perdre tout le fichier. Solution : le répliquer
		\item[-] il y a une limitation du nombre de pointeurs dans un bloc, ce qui limite la taille d'un fichier. Solutions : 
		
		\begin{itemize}
			\item chaînage de blocs
			\item arbre de blocs pour former un Btree
			\item structure combinée
		\end{itemize}				
		
	\end{itemize}
	
	Exemple d'indexage sur plusieurs niveaux.
	
	\dessin{76}
	
	Exemple de schéma combiné : UNIX, pour des blocs de 4K bytes. Une partie du fichier est accessible directement, puis une partie de la suite l'est par un single indirect, puis double, etc.
	
	\dessin{77}
	
\section{Gestion de l'espace libre}

	\subsection{Vecteur de bits}
	
	Pour $n$ blocs, un vecteur indique si un bloc $i$ est libre (avec la valeur 1) ou occupé (0). 
	
	Nombre de blocs nécessaires :
	
	$$\text{(nombre de bit par mot) } \times \text{ (nombre de mots de valeur 0) } + \text{ offset du premier bit à 1}$$
	
	\begin{itemize}
		\item[+] facilité d'avoir de l'espace continu
		\item[+] Le parcours du vecteur est rapide. Il suffit de trouver le premier bit différent de 1, or certains processeurs disposent d'une instruction spéciale qui permet de récupérer rapidement l'index du premier bit à 1 dans un byte.
		\item[-] besoin de plus d'espace
		\item[-] ce vecteur doit être en mémoire et sur un disque
	\end{itemize}
	
	Il faut veiller à synchroniser le bit map de la mémoire et celui du disque et surtout ne pas arriver à des situations où un bit vaut 1 en mémoire et 0 sur le disque, car en cas d'extinction brutale de la machine, le bloc sera marqué comme libre alors qu'il ne l'est pas, ce qui entraînera une perte de données.
	
	Solution :
	
	\begin{itemize}
		\item mettre le bit $i$ à 0 sur le disque
		\item allouer le bloc $i$
		\item mettre le bit $i$ à 0 en mémoire
	\end{itemize}
	
	\subsection{Liste liée}
	
	On utilise un système similaire que pour l'allocation de blocs ; une liste liée répertorie tous les blocs libres.
	
	\begin{itemize}
		\item[+] aucun gaspillage d'espace
		\item[-] difficile d'avoir de l'espace continu
	\end{itemize}
	
	Il est nécessaire de protéger le pointeur vers la liste.
	
	\subsection{Fichier spécial}
	
	Utilisation d'un fichier qui alloue tous les blocs libres.
	
	\begin{itemize}
		\item[-] nécessité d'une structure spéciale
	\end{itemize}
	
	\subsection{Grouping}
	
	Pour $n$ blocs libres, on garde un index ; un bloc pointe vers les $n - 1$ blocs restants. On garde alors une liste de tout ces blocs index.
	
	\subsection{Counting}
	
	C'est similaire à de la compression : on stocke le départ d'un espace libre continu et on y attache sa taille
	
\section{Efficacité et performances}

L'efficacité dépend

\begin{itemize}
	\item des algorithmes d'allocation du disque et de répertoire
	\item des types de données gardées dans les entrées des fichiers dans le répertoire
\end{itemize}

Les performances peuvent être augmentée grâce à

\begin{itemize}
	\item un cache du disque ; un espace de la mémoire serait réservé à la conservation de blocs qui sont souvent accédés.
	\item free-behind (on supprime une page du cache lorsque la prochaine page est demandée) et read-ahead (on met en cache plus d'une page) ; techniques pour optimiser l'accès séquentiel
	\item la réservation d'une partie de la RAM pour être un disque virtuel (ou RAM disk).
\end{itemize}

	\subsection{Page cache}
	
	Un page cache est un cache pour des pages plutôt que pour des blocs du disque et qui utilise des techniques de mémoire virtuelle. Les I/O mappées en mémoire utilise un page cache et une routine I/O utiliser un buffer (disk) cache.
	
	\dessinS{78}{0.3}
	
	Il est nécessaire d'utiliser un page cache car la mémoire est organisée en page et non en blocs. On pourrait n'avoir qu'un cache si un bloc du disque a la même taille qu'une page en RAM.

	\subsection{Unified Buffer Cache}
	
	Un buffer cache unifié utilise le même page cache pour cacher les pages mappées en mémoire et les I/O de systèmes de fichiers.
	
	\dessinS{79}{0.25}
	
\section{Log de systèmes de fichiers}

Un log enregistre toutes les mises à jour du système de fichiers comme des transactions. Une transaction est considérée comme exécutée (committed) une fois qu'elle a été écrite dans le log. Cependant, cela ne veut pas dire que le système de fichiers a été mis à jour ; les transactions du log sont écrites de manière asynchrone  et quand le système de fichiers est modifié, la transaction est retirée du log. Si le système crash, toutes les transactions du log devront toujours être exécutées.

Sans log, une modification peut ne pas être écrite sur le disque à cause du cache si le système crash. Avec le log, les modifications ne se font pas tout de suite sur le disque ; cet overhead très petit permet d'accélérer le tout.

\section{Stockage de masse}

Un disque-dur tourne à une vitesse de 5400 à 15000 tours/minute. Le taux de transfert est le débit auquel les données circulent entre le lecteur et l'ordinateur.

Le positioning time (random-access time) est le temps nécessaire pour déplacer le bras au cylindre désiré (seek time) et le temps pour que le secteur désiré tourne jusqu'à se trouver en dessous de la tête de lecture (rotational latency).

Un head crash se produit lorsque la tête du disque entre en contact avec le disque.

Les lecteurs sont attachés à l'ordinateur via un bus I/O : EIDE, ATA, SATA, USB, SCSI, etc. Un host controller dans l'ordinateur utilise ce bus pour dialoguer avec le contrôleur du disque (intégré dans le périphérique).

Un secteur est un bloc physique.

\dessinS{80}{0.3}

	\subsection{Structure de disque}
	
	Les lecteurs de disque sont adressés comme un tableau à une dimension de blocs logiques, où un bloc logique est la plus petite unité de transfert. Ce tableau est mappé dans les secteurs séquentiellement :
	
	\begin{itemize}
		\item le secteur 0 est le premier secteur de la première piste à la périphérie du cylindre.
		\item le mapping continue dans l'ordre à travers le track, puis le reste des tracks du cylindre, puis à travers les autres cylindres, de l'extérieur vers l'intérieur des plateaux.
	\end{itemize}
	
	\subsection{Attachement du disque}
	
	\subsubsection{Attachement local}
	Les hôtes de stockage sont accédés à travers des ports I/O en utilisant des bus I/O.
	
	Le SCSI est un bus qui peut compter jusqu'à 16 périphériques par câble. Un SCSI initiator lance une requête qui est servie par des SCSI targets. Chaque target peut avoir jusqu'à 8 unités logiques (disques attachés au contrôleur du périphérique).
	
	Le FC est une architecture sérielle à haute vitesse. [11.38]
	
	\subsubsection{NAS}
	
	Un NAS (network-attached storage) est un stockage disponible à travers un réseau plutôt qu'avec une connexion locale (telle qu'un bus). NFS et CIFS sont les protocoles généralement utilisés. L'implémentation se fait via des remote procedure calls (RPCs) entre l'hôte et le périphérique de stockage. Il y a également le protocole iSCSI qui utilise le protocole IP pour transporter le protocole SCSI.
	
	\dessinS{81}{0.3}
	
	\subsubsection{Réseau de stockage}
	
	Les réseaux de stockage (Storage Area Network) sont communs dans des environnements de stockage large. Plusieurs hôtes sont attachés à plusieurs matrices de stockage, ce qui permet une grande flexibilité.
	
	\dessinS{82}{0.3}
	
\section{Ordonnancement du disque}
	
L'OS est responsable de l'utilisation efficace du hardware. Pour les disques-durs, cela signifie qu'il faut un temps d'accès très court et de la bande passante.

Le temps d'accès possède deux composants majeurs :

\begin{itemize}
	\item seek time : temps pour que les têtes du disque se déplacent au-dessus du secteur désiré
	\item rotational latency : temps attendu pour que le disque tourne jusqu'à ce que les secteurs désirés soient en dessous de la tête de lecture
\end{itemize}

Le seek time est à minimiser, or il dépend très fortement de la seek distance. De ce fait, l'ordonnancement des tâches est crucial.

La bande-passante du disque est le nombre total de bytes transférés divisé par le temps entre la première requête pour un service et la complétion du dernier transfert.

Plusieurs algorithmes existent pour servir des requêtes I/O.

	\subsection{FCFS - First-Come First Served}
	
	\dessinS{83}{0.3}
	
	\subsection{SSTF - Shortest-Seek Time First}
	
	On sélectionne la requête qui engendra le seek time minimal à partir de la position courante de la tête. Il s'agit d'une forme d'ordonnancement SJF (Shortest Job First) et peut souffrir de starvation pour certaines requêtes.
		
	\dessinS{84}{0.3}
	
	\subsection{SCAN}
	
	Le bras du disque démarre à une extrémité du disque et se déplace jusqu'à l'autre extrémité, en servant les requêtes au passage.  Cet algorithme est souvent appelé elevator algorithm.
	
	\dessinS{85}{0.3}
	
	\subsection{C-SCAN}
	
	Cette version du SCAN possède un temps d'attente plus uniforme. La tête bouge d'une extrémité à l'autre du disque et sert les requêtes au passage. Lorsqu'il atteint une extrémité, elle retourne immédiatement au début du disque, sans servir de requête sur le retour.
	
	On traite ainsi les cylindres comme une liste circulaire qui s'étend du dernier cylindre au tout premier.
	
	\dessinS{86}{0.3}
	
	\subsection{C-LOOK}
	
	Il s'agit d'une version de C-SCAN où le bras va suffisamment loin que pour servir la dernière requête dans chaque direction, puis renverse sa direction sans atteindre la fin du disque.
	
	\dessinS{87}{0.3}
	
	\subsection{Choix d'un ordonnancement}
	
	\begin{itemize}
		\item SSTF est assez commun et naturel
		\item SCAN et C-SCAN sont meilleurs pour les systèmes qui ont une charge importante sur les disques
	\end{itemize}
	
	Les performances dépendent du nombre et du type de requête. Les requêtes pour des services du disque peuvent être influencées par la méthode d'allocation des fichiers. Gagner un peu sur le disque permet de beaucoup gagner en cycles CPU.
	
	L'algorithme d'ordonnancement du disque doit être écrit comme un module séparé de l'OS afin de pouvoir le remplacer par un algorithme différent si nécessaire.
	
	SSTF ou LOOK sont des choix raisonnables pour l'algorithme par défaut.
	
	
\section{Gestion des disques}

Un formatage de bas niveau (ou formatage physique) consiste à diviser le disque en secteurs dans lesquels le contrôleur peut lire et écrire.

Pour utiliser un disque pour y placer des fichiers, l'OS a besoin de stocker ses propres structures de données dessus :

\begin{itemize}
	\item une partition, qui divise le disque en un ou plusieurs groupes de cylindres
	\item un formatage logique permet de créer le système de fichier
\end{itemize}

Un bloc de boot initialise le système. Le bootstrap est stocké dans une ROM et pointe vers le bootstrap loader.

Des méthodes telles que sector sparing sont utilisées pour gérer les mauvais blocs.

\dessinS{88}{0.25}
	
\end{document}	